\section{Research Questions}

\begin{question}
Concerning the locus of $X_{59}$ over billiard 3-periodics (\cref{fig:05-x59-locus}, determine:
\begin{itemize}
\item An implicit and/or parametric equation;
\item The locations of its four self-intersections;
\item The $a/b$ such that if $X_{59}$ is on a self-intersections on the elliptic billiard minor axis, the 3-periodic is a right triangle? (it is close to $1.58$, see \cref{fig:05-x59-locus}(right).
\end{itemize}
\end{question}

\begin{question}
Prove that over billiard 3-periodics traversed continuously, the vertices of the extouch triangle, i.e., the 3 extouchpoints, will move in the same direction as 3-periodic vertices, whereas the Feuerbach point will move in the opposite direction.  
\end{question}

\begin{question}
Derive an expression (implicit and/or parametric) for the locus of $X_{26}$ in either the compact or non-compact case.
\end{question}

\begin{question}
Derive an expression of the non-elliptic locus of the vertices of the anticomplementary triangle over billiard 3-periodics. Show it is always external to the elliptic billiard. Derive its inflection points. See it \href{https://bit.ly/2RtUT00}{Live}.
\end{question}

\begin{question}
\label{que:05-x88-x162}
Derive an expression for $t$ where $X_{88}$ and $X_{162}$ are closest (there are 12 solutions). In \cref{fig:05-3d-torus}, the dashed meridian represents one such minimum which for $a/b=2$ occurs at $t{\simeq}41^\circ$. Notice it does not coincide with any critical points of motion.
\end{question}

\begin{question}
Show that the locus of the inversion of $X_1$ with respect to the moving circumcircle of billiard 3-periodics is also an ellipse. See it \href{https://bit.ly/3ujusan}{Live}.
\end{question}

\begin{question}
Show that the locus of the inversion of $X_3$ with respect to the moving incircle of billiard 3-periodics is also an ellipse. See it \href{https://bit.ly/2SwDLa4}{Live}.
\end{question}

\begin{question}
Prove \cref{prop:05-x100-x190}.
\end{question}

\begin{question}
Prove \cref{prop:05-x88-x162}, and derive $\alpha_{162}$ and $\alpha_{88}$. Numerically, these are approximately $1.164$ and $1.486$, respectively, see \cref{fig:05-x190-angular-velocity} (top right and bottom left).
\end{question}