The bicentric family is a 1d family of Poncelet N-gons interscribed between two specially-chosen circles \cite[Poncelet's Porism]{mw}. The special case of a family of triangles with fixed incircle and circumcircle was originally studied by Chapple 80 years before Poncelet \cite{odehnal2011-poristic}. Any pair of conics can be sent to a pair of circles via a suitable projective transformation \cite{akopyan2007-conics}. Based on this, in the 1820s Jacobi produced an alternative proof to Poncelet's Great theorem based on simplifications afforded by his elliptic functions over the bicentric family \cite{bos-1987,dragovic11,nash2018-poncelet}.

Referring to
\cref{fig:confocal}, a known fact is that the {\em polar image}\footnote{The polar of a point $P$ with respect to a circle $\C$ centered on $O$ is the line $L$ containing the inversion of $P$ wrt $\C$ and perpendicular to $OP$.} of two circles with respect to either one of their {\em limiting points}\footnote{A pair of circles is associated with a pair of {\em limiting points} $\ell_1,\ell_2$ which are centers of inversion that send the original circles to two distinct pairs of concentric circles \cite[Limiting Points]{mw}.} is a pair of confocal conics with a focus coinciding with the limiting point chosen \cite{akopyan2007-conics} (see \cref{app:polar-pedal}). Conversely, the bicentric family is the polar image of elliptic (or hyperbolic) billiard N-periodics with respect to a focus (see \cref{sec:five-polys}). Recall the latter conserve both perimeter\footnote{Billiard inscribed in hyperbolas conserve {\em signed} perimeter, see \cref{sec:five-polys}.} and Joachimsthal's constant \cite{sergei91}.

\begin{figure}
    \centering
    \includegraphics[width=.9\textwidth]{pics/0040_billiard_plot.pdf}
    \caption{The bicentric family  (solid orange) is the polar image of elliptic billiard N-periodics (blue) with respect to a circle (dashed gray) centered on $f_1$ (which coincides with limiting point $\ell_1$). Also shown are the constant-perimeter bicentric pedals (pink and purple) with respect to either limiting point, $f_1=\ell_1$ and $\ell_2$.  \href{https://youtu.be/8m21fCz8eX4}{Video}}
    \label{fig:confocal}
\end{figure}

%\begin{figure}
%    \centering
%    \includegraphics[width=.9\textwidth]{pics/0015_bicentric_with_concentric_pair.pdf}
%    \caption{A bicentric polygon (solid orange), interscribed between circles (dashed orange) centered on $O_{int}$ and $O_{ext}$. Also shown are the limiting points $\ell_1$ and $\ell_2$ as well as the two concentric pairs of circles (light blue) obtained by inverting the original circles wrt to unit circles centered at the the $\ell_i$. Note both concentric pairs have the same ratio of radii.
%    \label{fig:bicentric-with-concentric-pair}
%\end{figure}

\textbf{Main Results:}
Though the bicentric family was much studied in the last 200 years, interactive experimentation with their dynamic geometry has led us to detect and prove a few new curious facts, perhaps known to the giants of the XIX century but never jotted down.

\begin{itemize}
    \item \cref{thm:bicentric-sum}: The sum of the cosines of bicentric polygons is invariant over the family. This mirrors an invariant recently proved for elliptic billiard N-periodics \cite{reznik2020-intelligencer,garcia2020-new-properties,akopyan2020-invariants,bialy2020-invariants}.
    \item  \cref{thm:bicentric-pedal-perimeter} The perimeter of pedal polygons of the bicentrics with respect to its limiting points is invariant; see \cref{fig:confocal}. Notice this too mirrors perimeter invariance of elliptic billiard N-periodics.
    \item \cref{cor:inv-per}: Bicentric pedals with respect to a limiting point are identical to the inversion of billiard N-periodics with respect to a focus, therefore the latter also conserves perimeter. In fact it was this surprising observation (see this \href{https://youtu.be/wkstGKq5jOo}{Video}) that prompted the current article.
    \item \cref{conj:limiting-sum-cosines}: Experiments show that the two limiting pedal polygons also conserve their sum of cosines, except for  the case of the $N=4$ pedal with respect to $\ell_1$.
\end{itemize}



%\begin{figure}
%    \centering
%    \includegraphics[width=.9\textwidth]{pics/0020_two_pedals.pdf}
%    \caption{The two constant-perimeter pedal polygons (pink and purple) whose vertices are at the feet of perpendiculars dropped from $\ell_1$ and $\ell_2$ respectively, onto the sidelines of the bicentric family (solid orange).}
%    \label{fig:two-pedals}
%\end{f

%\begin{figure}
%    \centering
%    \includegraphics[width=.9\textwidth]{pics/0035_basic_billiard_plot.pdf}
%    \caption{The $f_1$-inversive polygon (pink) is the pedal polygon of a bicentric N-periodic (orange) with respect to the limiting point coinciding with $f_1$.}
%    \label{fig:basic-confocal}
%\end{figure}


%\begin{figure}
%    \centering
%    \includegraphics[width=.9\textwidth]{pics/0030_limacons.pdf}
%    \caption{Over the bicentric family (solid orange), the constant-perimeter $\ell_1$- and $\ell_2$ pedals (pink and purple) are inscribed in two separate Limaçons of Pascal: the former (green) is loopless, while the latter (aquamarine) has a loop passing through $\ell_2$.}
%    \label{fig:limacons}
%\end{figure}

\subsection*{Article Structure}
In \cref{sec:jacobi}, we review Jacobi's parametrization for bicentric polygons. We then use it to obtain expressions in terms of Jacobi elliptic functions for each of the above invariants, see \cref{sec:bicentric-sum-of-cosines,sec:pedal-perimeter}. \cref{sec:five-polys} paints a unified view of the five polygon families mentioned herein. A list of illustrative videos appear in \cref{sec:videos}. 

Details of polar and pedal transformations are covered in \cref{app:polar-pedal}. The parameters for a pair of confocal ellipses (or hyperbolas) which are the polar image of the bicentric pair are given in \cref{app:bicentric-to-confocal}. Conversely, the parameters for a bicentric pair which is the polar image of confocal ellipses are given in
\cref{app:confocal-to-bicentric}. In \cref{app:bicentric-vertices-n34} we provide elementary parametrizations for the vertices of $N=3$ and $N=4$ bicentric polygons. In \cref{app:pedal-perimeters-n34} we provide explicit expressions of their perimeters and sums of cosines as well as curious properties thereof.

%\begin{figure}
%    \centering
%    \includegraphics[width=.9\textwidth]{pics/0045_concentric_pairs_and_billiard.pdf}
%    \caption{The bicentric family (orange) is the polar image of billiard N-periodics (blue). These are Poncelet polygons interscribed between two confocal ellipses (black and brown). In particular one of the foci $f_1$ coincides with a first ``focal'' limiting point. In this configuration, the other one, $l_2$, appears in the positive x-axis. Also shown are the two pairs of concentric circles (aquamarine) centered on $C_1'$ and $C_2'$ which are inversive images of the bicentric circles wrt to unit radius circles centered at $f_1$ and $l_2$. Notice one of the pairs is centered at the billiard center, with a first (resp. second) circle circumscribing the billiard (resp. caustic).}
%    \label{fig:billiard-and-concentric-circles}
%\end{figure}


\subsection*{Related Work}

A few experimental of our experimental conjectures for elliptic billiard N-periodic invariants \cite{reznik2020-intelligencer,garcia2020-new-properties} have been proved: (i) invariant sum of cosines and (ii) invariant product of outer polygon cosines \cite{akopyan2020-invariants,bialy2020-invariants}, and (iii) invariant outer-to-orbit area ratio (for odd N) \cite{caliz2020-area-product}. Dozens of other conjectured invariants appear in \cite{reznik2021-fifty}.