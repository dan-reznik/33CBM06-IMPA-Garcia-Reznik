In the previous chapter we toured loci phenomena for billiard 3-periodics. Here we continue this exploration for other for 3-periodic families in other concentric, axis-aligned ellipse pairs.

\section{Summary}

\cref{tab:04-xn-comparison} summarizes the types of loci (point, circle, ellipse, etc.) for several triangle centers for all families mentioned above. These are organized within three groups A, B, and C with closely-related loci types. Exceptions are also indicated.

\begin{table}
\small
\centering
\begin{tabular}{|c||c|c|c||c|c|c||c|c||}
\hline
& \multicolumn{3}{c||}{Group A} & \multicolumn{3}{c||}{Group B} & \multicolumn{2}{c||}{Group C} \\
\hline
 & Conf. & F.I & Por. & \makecell{Conf.\\Exc} & F.II &  \makecell{Por.\\Exc.} & F.III & Broc. \\
 \hline
$X_1$ & E & P & P & X&X & X & 4 & X \\
$X_2$ & E & E & C & E&C & P & P & C \\
$X_3$ & E & C & P & E&P & P & E & P \\
$X_4$ & E & E & C & E&C & P & E & C \\
$X_5$ & E & C & C & E&C & P & E & C \\
$X_6$ & 4 & 4 & \textcolor{red}{\textbf{E}} & P&E & C & E & P \\
$X_7$ & E & E & C & X & X & X & X & X \\
$X_8$ & E & E & C & X & X & X & X & X \\
$X_9$ & P & E & C & X & X & X & X & X \\
$X_{10}$ & E & E & C & X & X & X & X & X \\
$X_{11}$ & E$''$ & C$''$ & C$''$ & X & X & \textcolor{red}{\textbf{C$_5$}} & X & X \\
$X_{12}$ & E & C & C & X & X & X & X & X \\
$X_{13}$ & X & X & X & X & X & X & C & C \\
$X_{14}$ & X & X & X & X & X & X & C & C \\
$X_{15}$ & X & X & X & X & X & X & C & P \\
$X_{16}$ & X & X & X & X & X & X & C & P \\
$X_{99}$ & X & X & C$'$ & \textcolor{red}{\textbf{X}}&C$'$ & C$'$ & E$'$ & C$'$ \\
$X_{100}$ & E$'$ & E$'$ & C$'$ & \textcolor{red}{\textbf{X}}&C$'$ & C$'$ & \textcolor{red}{\textbf{X}} & C$'$ \\
$X_{110}$ & X & X & C$'$ & E$'$&C$'$ & C$'$ & \textbf{X} & C$'$ \\
 \hline
\end{tabular}
\caption{Types of loci for several triangle centers over several Poncelet triangle families, divided in 3 groups A,B,C with closely-related metric phenomena: (i) confocal, fam. I, poristics; (ii) confocal excentral, fam. II, poristic excentral triangles; (iii) fam. III and Brocard porism. Symbols P, C, E, and X indicate point, circle, ellipse, and non-elliptic (degree not yet derived) loci, respectively. A number refers to the degree of the non-elliptic implicit, e.g., '4' for quartic. A singly (resp. doubly) primed letter indicates a perfect match with the outer (resp. inner) conic in the pair. The symbol C$_5$ refers to the nine-point circle. The boldface entries indicate a discrepancy in the group (see text). Note: $X_n$ for the confocal and poristic excentral triangles refer to triangle centers of the family itself (not of their reference triangles).}
\label{tab:04-xn-comparison}
\end{table}

The first row reveals that out of the 8 families considered only in the confocal case is the locus of the incenter $X_1$ an ellipse. Additionaly experimentation has suggested an intriguing conjecture:

The plethora of circles in the poristic family had already been shown in \cite{odehnal2011-poristic}. An above-than-expected frequency of ellipses for the confocal pair was signalled in \cite{garcia2020-ellipses}. As mentioned above, irrational centers $X_k$, $k\in[13,16]$ sweep out circles for the homothetic pair. $X_{15}$ and $X_{16}$ are known to be stationary over the Brocard family \cite{bradley2007-brocard}, however the locus of $X_{13}$ and $X_{14}$ are circles! Also noticeable is the fact that (i) though in the confocal pair the loci of $X_1$ and $X_6$ are an ellipse and a quartic, respectively, in both family II and family III said locus types are swapped.

It is well-known that there is a projective transformation that takes any Poncelet family to the the confocal pair,  \cite{dragovic11}. In this case only  projective properties are preserved. 
If one restricts the set of possible transformations to either affine ones or similarities (which include rigid transformations), one can construct the two-clique graph of interrelations shown in \cref{fig:03-transformations}.

As mentioned above, the confocal family is the affine image of either family I or family II. In the first (resp. second) case the caustic (resp. outer ellipse) is sent to a circle. Though the affine group is non-conformal, we showed above that both families conserve the sum of cosines. One way to see this is that there is an alternate, conformal path which takes family I triangles to the confocal ones, namely a rigid rotation (yielding poristic triangles), followed by a variable similarity (yielding the confocal family).

A similar argument is valid for family II triangles: there is an affine path (non-conformal) to the confocal family though both conserve the product of cosines. Notice an alternate conformal composition of rotation (yielding poristic excentral triangles) and a variable similarity (yielding confocal excentral triangles). All in this path conserve the product of cosines.

Finally, family III and Brocard porism triangles form an isolated clique. As mentioned in \cite{reznik2020-similarityII}, these are variable similarity images of one another but cannot be mappable to the other families via similarities nor affinely.



