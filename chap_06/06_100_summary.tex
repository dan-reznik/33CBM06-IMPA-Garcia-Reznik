\section{Summary}

\cref{tab:04-xn-comparison} summarizes the types of loci (point, circle, ellipse, etc.) for some triangle centers for families analyzed in this and previous chapters (including non-concentric such as poristic triangles and Brocard's porism). Families are grouped according to similar patterns in their loci types.

\begin{table}
\small
\centering
\begin{tabular}{|c||c|c|c||c|c|c||c|c||}
\hline
& \multicolumn{3}{c||}{Group A} & \multicolumn{3}{c||}{Group B} & \multicolumn{2}{c||}{Group C} \\
\hline
 & Conf. & Inc. & Por. & Exc. & Circ. &  \makecell{Por.\\Exc.} & Hom. & Broc. \\
 \hline
$X_1$ & E & P & P & X&X & X & 4 & X \\
$X_2$ & E & E & C & E&C & P & P & C \\
$X_3$ & E & C & P & E&P & P & E & P \\
$X_4$ & E & E & C & E&C & P & E & C \\
$X_5$ & E & C & C & E&C & P & E & C \\
$X_6$ & 4 & 4 & \textcolor{red}{\textbf{E}} & P&E & C & E & P \\
$X_7$ & E & E & C & X & X & X & X & X \\
$X_8$ & E & E & C & X & X & X & X & X \\
$X_9$ & P & E & C & X & X & X & X & X \\
$X_{10}$ & E & E & C & X & X & X & X & X \\
$X_{11}$ & E$''$ & C$''$ & C$''$ & X & X & \textcolor{red}{\textbf{C$_5$}} & X & X \\
$X_{12}$ & E & C & C & X & X & X & X & X \\
$X_{13}$ & X & X & X & X & X & X & C & C \\
$X_{14}$ & X & X & X & X & X & X & C & C \\
$X_{15}$ & X & X & X & X & X & X & C & P \\
$X_{16}$ & X & X & X & X & X & X & C & P \\
$X_{99}$ & X & X & C$'$ & \textcolor{red}{\textbf{X}}&C$'$ & C$'$ & E$'$ & C$'$ \\
$X_{100}$ & E$'$ & E$'$ & C$'$ & \textcolor{red}{\textbf{X}}&C$'$ & C$'$ & \textcolor{red}{\textbf{X}} & C$'$ \\
$X_{110}$ & X & X & C$'$ & E$'$&C$'$ & C$'$ & \textbf{X} & C$'$ \\
 \hline
\end{tabular}
\caption{Loci types (P, C, E, X indicate point, circle, ellipse, and non-elliptic (degree not yet derived) loci, respectively) of some triangle centers over 3-periodic families. These are clustered in in 3 groups A,B,C sharing many metric phenomena: (i) confocal, incircle, poristic; (ii) excentral, circumcircle, poristic-excentral; (iii) homothetic and Brocard porism. A numeric entry indicates the degree of the non-elliptic implicit, e.g., '4' for quartic. A singly (resp. doubly) primed letter indicates a perfect match with the outer (resp. inner) conic in the pair. The symbol C$_5$ refers to the nine-point circle. The boldface entries indicate a discrepancy in the cluster. Note: $X_n$ for the confocal and poristic excentral triangles refer to triangle centers of the family itself (not of their reference triangles).}
\label{tab:04-xn-comparison}
\end{table}

The first row reveals that out of the 8 families considered only in the confocal case is the locus of the incenter $X_1$ an ellipse, suggesting this is a rare phenomenon.

The plethora of circles in the poristic family had already been shown in \cite{odehnal2011-poristic}. A significant occurrence of ellipses in the confocal pair was signalled in \cite{garcia2020-ellipses}. As mentioned above, irrational centers $X_k$, $k\in[13,16]$ sweep out circles for the homothetic pair. $X_{15}$ and $X_{16}$ are known to be stationary over the Brocard family \cite{bradley2007-brocard}, however the locus of $X_{13}$ and $X_{14}$ are circles. Also noticeable is the fact that (i) though in the confocal pair the locus of $X_1$ (resp. $X_6$) is an ellipse (resp. quartic), locus types are swapped for both circumcircle and homothetic families.

It is well-known that there is a projective transformation that takes any Poncelet family to the the confocal pair,  \cite{dragovic11}. In this case only projective properties are preserved.

As mentioned above, the confocal family is the affine image of either the incircle or circumcircle family. In the first (resp. second) case the caustic (resp. outer ellipse) is sent to a circle. Though the affine group is non-conformal, we showed above that both families conserve the sum of cosines. One way to see this is that there is an alternate, conformal path which takes incircle 3-periodics to confocal ones, namely through a rigid rotation (yielding poristic triangles), followed by a variable similarity (yielding the confocal family).

A similar argument is valid for circumcircle triangles: there is an affine path (non-conformal) to the confocal family though both conserve the product of cosines. Notice there is also an alternate conformal composition of rotation (yielding poristic excentral triangles) and a variable similarity (yielding confocal excentral triangles). All in this path conserve the product of cosines.

Finally, homothetic and Brocard porism 3-periodics form an isolated clique. As mentioned in \cite{reznik2020-similarityII}, these are variable similarity images of one another but cannot be mappable to the other families via similarities nor affinely.

\subsection{Loci Types, CAP families}
\label{sec:06-loci-types}

Below we list triangle centers such that their loci types are either points or conics. These are obtained via numerical simulation amongst the first 200 centers in \cite{etc}.

\begin{itemize}
    \item Confocal (stationary $X_9$)
    \begin{itemize}
	\item Circles (0): n/a
	\item Ellipses (42): 1, 2, 3, 4, 5, 7, 8, 10, 11, 12, 20, 21, 35, 36, 40, 46, 55, 56, 57, 63, 65, 72, 78, 79, 80, 84, 88, 90, 100, 104, 119, 140, 142, 144, 145, 149, 153, 162, 165, 190, 191, 200. Note: the first 29 in the list were proved in \cite{garcia2020-ellipses}
	\end{itemize}
    \item Incircle: (stationary $X_1$)
    \begin{itemize}
    \item Circles (15): 3, 5, 11, 12, 35, 36, 40, 46, 55, 56, 57, 65, 80, 119, 165.
    \item Ellipses (26): 2, 4, 7, 8, 9, 10, 20, 21, 63, 72, 78, 79, 84, 90, 100, 104, 140, 142, 144, 145, 149, 153, 170, 176, 191, 200.
    \end{itemize}
    \item Circumcircle: (stationary $X_3$)
	\begin{itemize}
	\item Circles (28): 2, 4, 5, 20, 22, 23, 24, 25, 26, 74, 98, 99, 100, 101, 102, 103, 104, 105, 106, 107, 108, 109, 110, 111, 112, 140, 156, 186, 201.
	\item Ellipses (34): 6, 15, 21, 27, 28, 39, 49, 51, 52, 54, 58, 61, 64, 66, 67, 68, 69, 70, 113, 125, 141, 143, 146, 154, 155,  159, 161, 182, 184, 185, 193, 195, 199.
	\end{itemize}
    \item Homothetic: (stationary $X_2$)
	\begin{itemize}
    \item Circles (4): 13, 14, 15, 16.
	\item Ellipses (29): 3, 4, 5, 6, 17, 18, 20, 32, 39, 61, 62, 69, 76, 83, 98, 99, 114, 115, 140, 141, 147, 148, 182, 187, 190, 193, 194.
	\end{itemize}
	\item Dual: (stationary: $X_4$)
	\begin{itemize}
    \item Circles (0): n/a
    \item Ellipses (10): 2, 3, 5, 20, 64, 107, 122, 133, 140, 154.
    \end{itemize}
	\item Excentral: (stationary: $X_6$)
	\begin{itemize}
    \item Circles (0): n/a
    \item Ellipses (39): 2, 3, 4, 5, 20, 22, 23, 24, 25, 26, 49, 51, 52, 54, 64, 66, 67, 68, 69, 70, 74, 110, 113, 125, 140, 141, 143, 146, 154, 155, 156, 159, 161, 182, 184, 185, 186, 193, 195.

    \end{itemize}
\end{itemize}

Semi-axes lengths for the elliptic loci of many triangle centers are available in \cite{garcia2021-ellipses-web}.

\subsection{Loci Types, NCAP Families}
\label{sec:06-ncap-loci}

For completeness, included below are point and/or conic loci for both Poristic and Brocard triangles. These include many stationary centers as well as segment and hyperbolic loci. 

\begin{itemize}
\item Poristic, see \cite{odehnal2011-poristic}
    \begin{itemize}
    \item Points (11): 1, 3, 35, 36, 40, 46, 55, 56, 57, 65, 165. 
    \item Segments (2): 44, 171.
    \item Circles (46): 2, 4, 5, 7, 8, 9, 10, 11, 12, 20, 21, 23, 63, 72, 74, 78, 79, 80, 84, 90, 98, 99, 100, 101, 102, 103, 104, 105, 106, 107, 108, 109, 110, 111, 112, 119, 140, 142, 143, 144, 145, 149, 153, 186, 191, 200.
	\item Ellipses (39): 6, 19, 22, 24, 25, 28, 31, 33, 34, 37, 38, 41, 42, 43, 45, 47, 48, 51, 52, 54, 58, 59, 60, 71, 73, 77, 81, 88, 89, 169, 170, 181, 182, 184, 185, 195, 197, 198, 199.
	\item Hyperbolas (7): 26, 49, 64, 154, 155, 156, 196.
	\end{itemize}
\item{Brocard porism}
    \begin{itemize}
    \item Points (10): 3, 6, 15, 16, 32, 39, 61, 62, 182, 187.
    \item Segments (3): 50, 52, 58.
	\item Circles (38): 2, 4, 5, 13, 14, 17, 18, 20, 23, 69, 74, 76, 83, 98, 99, 100, 101, 102, 103, 104, 105, 106, 107, 108, 109, 110, 111, 112, 114, 115, 140, 141, 147, 148, 183, 186, 193, 194.
	\item Ellipses (6): 24, 25, 51, 143, 157, 18.
	\item Hyperbolas (5): 26, 49, 64, 154, 159.
	\end{itemize}
\end{itemize}


