\section{Exercises}

\begin{exercise}
Prove that over incircle 3-periodics, the power of the center with respect to the (fixed radius) circumcircle is invariant and  equal to $-a b$.
\end{exercise}

\begin{exercise}
Compute $a/b$ of the external ellipse in the incircle CAP family such that (i) the circular locus of $X_3$ coincides with the incircle, (ii) the elliptic locus of $X_4$ touches the outer ellipse at its top and bottom vertices, and (iii) the circular locus of $X_5$ coincides with the incircle. See it \href{https://bit.ly/3wzw1CD}{Live1}, \href{https://bit.ly/3vnVHlL}{Live2}.
\end{exercise}

\begin{exercise}
Derive the radius of the circumcircle in the same-named family such that the quartic locus of $X_1$ and the circular locus of $X_4$ intersect at four points on the inner ellipse, see it \href{https://bit.ly/3wzQkjp}{Live}. 
\end{exercise}

\begin{exercise}
Prove \cref{prop:06-homoth-four-circles}.
\end{exercise}

\begin{exercise}
Prove that over homothetic 3-periodics, the radius of the circular locus of $X_{16}$ is minimum when $a/b=3$.
\end{exercise}

\begin{exercise}
Prove that at $a/b=\sqrt{5}$, the elliptic loci of the Brocard points over homothetic 3-periodics are internally tangent to the inner ellipse.  See it \href{https://bit.ly/3fkEMuG}{Live}.
\end{exercise}

\begin{exercise}
Derive the $a/b$ such that the elliptic loci of the Brocard points over the homothetic family intersect the y axis at $b/2$, i.e., at the top vertex of the caustic. See it \href{https://bit.ly/3fm8G1u}{Live}.
\end{exercise}

\begin{exercise}
Prove that over homothetic 3-periodics, the locus of the Brocard midpoint $X_{39}$ is an ellipse, derive its axis.
\end{exercise}

\begin{exercise}
Show that over homothetic 3-periodics, the elliptic locus of the vertices of the first Brocard triangle is interior to the inner ellipse.
\end{exercise}

\begin{exercise}
Compute the invariant similarity ratio of homothetic 3-periodics to the first Brocard triangles. 
\end{exercise}

\begin{exercise}
Derive expressions for the areas in \cref{cor:06-equi-areas}.
\end{exercise}

\begin{exercise}
Synthesize a triangle center such that over billiard 3-periodics its locus is a circle? Hint: it will be an affine combination of $X_2$ and $X_3$.
\end{exercise}

\begin{exercise}
Derive semi-axes for the dual family elliptic loci of $X_2$, $X_3$, and $X_5$ in \cref{prop:06-dual-x2345}.
\end{exercise}

\begin{exercise}
As shown in \cref{sec:06-ncap-loci}, over poristic triangles, the locus of $X_{44}$, and $X_{171}$ are segments. Derive their data. Do the same for the segment-loci of $X_{50}$, $X_{52}$, $X_{58}$ over Brocard porism 3-periodics.
\end{exercise}

\begin{exercise}
Given an ellipse $\E$ with semi-axes $a$, $b$, consider a non-Ponceletian family of triangles with two vertices fixed on the foci of $\E$ and a third one which sweeps the boundary. Show the locus of the incenter of this family is an ellipse. See it \href{https://bit.ly/3up4a6V}{Live}.
\end{exercise}

\begin{exercise}
Prove that over the Brocard porism, the locus of $X_{114}$ is a circle concentric with, and exterior to, the Brocard inellipse. Derive its radius. \href{https://bit.ly/3g6pmcv}{Live}
\end{exercise}

\begin{exercise}
Prove that over the Brocard porism, the locus of $X_{115}$ is a circle concentric with the Brocard inellipse of radius equal to the latter's minor semi-axis. \href{https://bit.ly/3pfR5Mc}{Live}
\end{exercise}

\begin{exercise}
Over the Brocard porism, the locus of $X_{185}$ is an ellipse which intersects the major axis of the Brocard inellipse $\E'$ in two points $A$ and $B$, see it \href{https://bit.ly/2Rn0chC}{Live}. In the $1<a/b<2$ range, $A,B$ appear to lie between the foci of $\E'$, however for larger $a/b$, e.g., $a/b=3$, the locus seems to pass through the foci, see it \href{https://bit.ly/3icu7Eh}{Live}. Prove or disprove this statement. Derive the center and semi-axes of the locus. 
\end{exercise}