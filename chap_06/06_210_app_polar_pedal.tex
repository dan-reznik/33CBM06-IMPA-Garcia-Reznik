
We review properties of polar and pedal transformations. A detailed treatment is found in \cite{akopyan2007-conics,stachel2019-conics}.

In the discussion that follows, all geometric objects are contained in a fixed plane. Let $\C$ be a circle centered at $f_1$. The polar transformation with respect to $\C$ maps each straight line not passing through $f_1$ into a point, and maps each point different from $f_1$ into a straight line. This is done in the following manner.

 Let $p\neq f_{1}$ be a point and let ${p}^\dagger$ be the inversion of $p$ with respect to $\C$. The straight line $L_{p}$ that passes through ${p}^\dagger$ and is orthogonal to the line joining $p$ and ${p}^\dagger$ is the polar of $p$ with respect to $\C$. Conversely, a line $L$ not passing through $f_1$ has a point $q$ as its pole with respect to $\C$ if $L=L_{q}$.

For a smooth curve $\gamma$ not passing through $f_1$, we can define the polar curve $\gamma^{\star}$ in two equivalent ways. Let $p$ be a point of $\gamma$ and $T_{p}\gamma$ the tangent line to $\gamma$ at $p$, we define $p^{\star}=(T_{p}\gamma)^{\star}$, and $\gamma^{\star}$ is the curve generated by $p^{\star}$ as $p$ varies along $\gamma$. We can also think of $\gamma^{\star}$ as the curve that is the envelope of the 1-parameter family of lines $L_{p}$, where $p$ is a point of $\gamma$.

The notion of a polar curve can be naturally extended to polygons in the following manner: let $L_{j}$, $j=1,2,...,N$ be the consecutive sides of a planar polygon $\P$, and let $q_{j}$ be the corresponding poles, then this indexed set of points are the vertices of what we call the polar polygon $\P^{\star}$. Alternatively, we can consider the polars of vertices of $\P$, and their consecutive intersections do define the vertices of $\P^{\star}$.


Although the next results are certainly classical, we couldn't find them explicitly in the literature, so we include them for the reader's convenience.

\begin{lemma}
Let $\E$ be an ellipse and $f_1$ one its the foci. Then the polar curve $\E^{\star}$ with respect to a circle $\C$ centered at $f_1$ is a circle. Let $\H$ be a hyperbola and $f_1$ one of its foci. Then the polar curve $\H^{\star}$ with respect to a circle $\C$ centered at $f_1$ is a circle minus two points.
\label{lem:polar}
\end{lemma}
\begin{proof}

We will use polar coordinates for our computations. Without loss of generality, let $f_{1}=(0,0)$ and consider the parametrized conic given by:

\[ \gamma(t)=\left[\frac{a(1-e^2)}{1+e\cos{t}}\cos{t}, \frac{a(1-e^2)}{1+e\cos{t}}\sin{t}\right] \]

\noindent where, if $e>1$ the trace of $\gamma$ is a hyperbola and if $e<1$ the trace of $\gamma$ is an ellipse. The expression for the polar curve $\gamma^{\star}(t)$ is obtained by direct computation: compute the unit normal $\mathbf{n}(t)$ to $\gamma(t)$, and  the distance $d(t)$ from the tangent line through $\gamma(t)$ to $f_{1}$. This yields:
 
\[ \gamma^{\star}(t)=\left[\frac{e+\cos{t}}{a(1-e^2)},\frac{\sin{t}}{a(1-e^2)}\right] \]

\noindent whose trace is clearly contained in a circle. For the hyperbola, the parameter $t$ is such that $1+e\cos(t) \neq 0 $, this is why $\H^{\star}$ is a circle minus two points.
  
\end{proof}     


\begin{lemma}
\label{lem:limit-focus}
Let $\E_{1}$ and $\E_{2}$ be two confocal ellipses and $\E_{1}^{\star}$ and $\E_{2}^{\star}$ be the circles as in Lemma \ref{lem:polar}, then $f_{1}$ is a limiting point of the pencil of circles defined by $\E_{1}^{\star}$ and $\E_{2}^{\star}$. In a similar way, let $\E$ and $\H$ be respectively an ellipse and hyperbola that are confocal, and let $\E^{\star}$ and $\H^{\star}$ be the circle and the circle minus 2 points, as in Lemma \ref{lem:polar}, then $f_{1}$ is a limiting point of the pencil of circles defined by $\E^{\star}$ and the circle that contains $\H^{\star}$.
\end{lemma}

\begin{proof}
Given two circles $\C_{1}$ and $\C_{2}$, a classical result states that the limiting points $\delta_{\pm}$ of the pencil of circles determined by $\C_{1}$ and $\C_{2}$ are such that the inversion of $\C_{1}$ and $\C_{2}$ with respect to circles centered on $\delta_{\pm}$ are concentric.

If we denote by $a_{i}$ and $e_{i}$, $i=1,2$, respectively, the semi-major diameter and eccentricity of the ellipses $\E_{1}$ and $\E_{2}$, then, by symmetry, we can define an unknown limiting point as $\delta_{p}=(x,0)$, and the concentric circle condition then becomes a quadratic equation in the variable $x$, where the coefficients depend on $a_{1}$, $a_{2}$, $e_{1}$ and $e_{2}$. Using the fact that $\E_{1}$ and $\E_{2}$ are confocal, which is equivalent to $e_{1}a_{1}=e_{2}a_{2}$, and with some algebraic manipulations, the quadratic can be written as:

\[ (a_{1}-a_{2})(a_{1}+a_{2})(e_{2}a_{2}x+1)=0 \]

\noindent so $f_{1}=(0,0)$ is indeed one of the limiting points of the pencil of circles.
\end{proof}  

