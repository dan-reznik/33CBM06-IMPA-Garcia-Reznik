\begin{exercise}\label{ex:1chap1}
Let $P=[p:q:r] $ be an interior point of an equilateral triangle $T=ABC$.

\noindent i) Show that the distances of $P$ to the sidelines of the triangle $T$ are given by
\[ [pk,qk,rk], \;\;\;k=\frac{2\Delta}{pa+qb+rc}\]
where $\Delta$ is the area of the triangle.

 \noindent ii)
 Show that  $k(p+q+r)$ is equal to the length of the triangle's altitude, i.e., the sum is independent of the position of the point.
 
 \noindent iii) Show that the same result is true when $P$ is an exterior point, but here we need to consider distances with signal, according to the position of $P$. See \cref{fig:trilinear_signal}.
 
 \noindent iv) Show that the reciprocal is true, i.e., if the sum of distances is independent of the point, then the triangle is equilateral.
 
 \noindent iv) Generalize items ii), iii) and iv)  for regular polygons.
 \end{exercise}
 
 \begin{exercise}\label{ex:2chap1} Show that $X_i$??  of the reference triangle    is the symmedian point of its excentral triangle.
 
  \end{exercise}
 
  \begin{exercise}\label{ex:3appA}
 Let $ABC$ be a  triangle and    $P$ and $Q$ be isogonal conjugates.  Then the circumcenters of the triangles $BPC$ and $BQC$ are inverses with respect to the circumcircle of the triangle $ABC$.
   \end{exercise}
  
   \begin{exercise}\label{ex:4appA}
   Let $P$ and $Q$ be isogonal conjugates in the interior of
the triangle $ABC$. Then the   pedal  triangles of $P$ and $Q$
 share a circumcircle. Moreover, the center of this circle is the midpoint
of $PQ$.
    \end{exercise}
   
      \begin{exercise}\label{ex:5app}
  Let $P $ and $Q$ isogonal conjugates.  If the point
is reflected about the sides
$AB$, $BC$ and $AC$.
  Then the resulting triangle has circumcenter the point $P$.
 
    \end{exercise}  \begin{exercise}\label{ex:6app}
  Let $\mathcal{E}$ be an ellipse inscribed in a triangle $ABC$. Then foci $F_1$ and $F_2$ of $\mathcal{E}$ are isogonal conjugates. 
 
    \end{exercise}