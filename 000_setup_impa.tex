  %%% NAO TROCAR ESTA ORDEM!
% \usepackage[T2A]{fontenc}
%\usepackage[russian,english]{babel}
% \usepackage[utf8]{inputenc}
 \usepackage{lmodern}
%\usepackage{polyglossia}
%  \setdefaultlanguage{english}
  %\setotherlanguages{portuges,french,german,latin,greek}
%\usepackage{csquotes}

% Medidas mais próximas das finais:
\usepackage{geometry}
% KDP Print
\geometry{paperwidth=6.125in,
          paperheight=9.25in,
          margin=1.75cm,
          top=1.53cm,
          bottom=1.53cm,
          includehead,
          includefoot}



\usepackage{polyglossia}
\setmainlanguage{english}
\setotherlanguage{russian}

\newfontfamily\russianfont[Script=Cyrillic]{Linux Libertine Display O}
\newcommand{\Sha}{\mathord{\textit{\textrussian{ш}}}}


\usepackage{amsmath, latexsym, amssymb,amsbsy, amsthm,amsfonts,amscd,mathtext,float,makecell}
\usepackage{graphicx,wrapfig,graphics}
%\usepackage[T1]{fontenc}  
\usepackage[dvipsnames]{xcolor}
%\usepackage[russian,english]{babel}
%\usepackage{titlepic}

\usepackage{csquotes}
%\usepackage{polyglossia}
%\setmainlanguage{english}
%\setotherlanguage{russian}

%\setmainfont{FreeSerif}
%\setsansfont{FreeSans}
%\setmonofont{FreeMono}
 %\usepackage{svg}

% Antigo estilo do Dan:
%\usepackage[style=chicago-authordate,sorting=ynt]{biblatex}

% Bibliographical citations:
\usepackage[style=authoryear,
            sorting=nyt,
            backref=true,
            backrefstyle=three,
            isbn=false, backend=biber]{biblatex}
\let\cite\textcite

%\usepackage{imakeidx}
%  \makeindex[name=not, intoc=true, columns=2, program=texindy, options=-L english -C utf8, title=Index of Notation]
%  \makeindex[name=aut, intoc=true, columns=2, program=texindy, options=-L english -C utf8, title=Index of Authors]
%  \makeindex[name=std, intoc=true, columns=2, program=texindy, options=-L english -C utf8, title=Index]

% Use the package {nomencl} if you prefer the definitions together with the symbols.
%\usepackage[refpage,intoc]{nomencl}
%  \makenomenclature
%  \renewcommand{\nomname}{List of Symbols}

% For linkig it all.
\usepackage[psdextra,
            colorlinks=true,
            linkcolor=blue,
            citecolor=blue
            ]{hyperref}

% To refer to fórmulas, theorems, chapters, etc ... 
\usepackage[nameinlink,capitalize,noabbrev]{cleveref}

% Declare some macros to use in the text:
\DeclareMathOperator{\Div}{div}
\DeclareMathOperator{\Rot}{rot}
\newtheorem{step}{Step}
\newtheorem{theorem}{Theorem}
\newtheorem*{theorem*}{Theorem}
\newtheorem{observation}{Observation}
\newtheorem{proposition}{Proposition}
\newtheorem{conjecture}{Conjecture}
\newtheorem{corollary}{Corollary}
\newtheorem{property}{Property}
\newtheorem{exercise}{Exercise}
\newtheorem{lemma}{Lemma}
%\theoremstyle{remark}
\newtheorem{remark}{Remark}
%\theoremstyle{definition}
\newtheorem{definition}{Definition}
\newtheorem{question}{Question}
\newtheorem{example}{Example}
\newcommand{\Cm}{\mathcal{C}}
\newcommand{\B}{\mathcal{B}}
\newcommand{\K}{\mathcal{K}}
\newcommand{\E}{\mathcal{E}}
\newcommand{\Hc}{\mathcal{H}}
\newcommand{\Tm}{\mathcal{T}}
\newcommand{\Om}{\mathcal{O}}
\newcommand{\F}{\mathcal{F}}
\renewcommand{\P}{\mathcal{P}}
\newcommand{\D}{\mathbb{D}}
\newcommand{\T}{\mathbb{T}}
\newcommand{\R}{\mathbb{R}}
\newcommand{\Cp}{\mathbb{C}}
\newcommand{\ol}{\overline}
\renewcommand{\l}{\lambda}
\newcommand{\X}{\mathcal{X}}
\newcommand{\ab}{_{\alpha,\beta}}
\def\Jcn{\mathrm{JacobiCN}}
\def\Jsn{\mathrm{JacobiSN}}
\def\cn{\mathrm{cn}}
\def\sn{\mathrm{sn}}
\def\am{\mathrm{am}}
\def\dn{\mathrm{dn}}

%%%

%\usepackage{lineno}
\def\linenumberfont{\normalfont\small\sffamily}
%a4: 210 x 297
%\textwidth=125mm
%\textheight=195mm
%\arraycolsep=2pt
%\captionsetup{width=120mm}

\usepackage{comment}
\usepackage{microtype}
\usepackage{footnote}
\newcommand{\Mod}[1]{\ (\mathrm{mod}\ #1)}
 
\newcommand{\torp}[2]{\texorpdfstring{#1}{#2}}


%\newcommand{\Sha}{\mathord{\textit{ш}}}
%\newcommand{\Zhe}{\mathord{\textit{Ж}}}

\graphicspath{{pics_tex/},{chap_01/pics/},{chap_02/pics/},{chap_03/pics/},{chap_04/pics/},{chap_05/pics/},{chap_06/pics/},{chap_07/pics/},{chap_08/pics/},{chap_09/pics/},{zappA/pics/}, {zappB/pics/}, {zappC/pics/},{zappD/pics/}}

%\usepackage{tikz}
%\include{000_setup_tikz}

