

%\section{Elliptic Billiard}

 

 \begin{theorem} Consider an  elliptic billiard defined  in the ellipse $\mathcal{E}$ given by $x^2/a^2+y^2/b^2=1$ $(a>b)$. Let $F_1=(-c,0)$ and $F_2=(c,0) $ the foci of $\mathcal{E}$. Let $(P_n)=(P_n)_{n\in\mathbb{Z}}$ be a billiard orbit inscribed in $\mathcal{E}$. Then:
 
 \begin{itemize} 
 \item[i)] If the segment of orbit $P_0P_1$ is outside the segment $F_1F_2$ then the caustic of the orbit $(P_n)$ is a confocal ellipse $\mathcal{E}_1$ and the orbit is periodic or dense in the annulus defined by the pair  $\{\mathcal{E},   \mathcal{E}_1 \}$.
 
  \item[ii)] If the segment of orbit $P_0P_1$   intersects the segment $F_1F_2$ then the caustic of the orbit is a confocal hyperbola  $\mathcal{H}_1$ and the orbit is periodic or dense in the disk defined by the ellipse $\mathcal{E}$ and the caustic $\mathcal{H}_1$.
  
   \item[iii)] If the segment of orbit $P_0P_1$ pass through a focus  then the orbit pass through the other focus and is asymptotic to the 2-periodic orbit (diameter of the ellipse $\mathcal{E}$) in the past (backward) and  the future(forward).
\end{itemize}
 \begin{figure}[H]
	\begin{center}
		\def\svgwidth{1.0\textwidth}
		\input{pics_tex/caustica1.pdf_tex}
		%\includegraphics[angle=0, width=12cm]{bi.pdf}
		\caption { Three types of billiard orbits in the ellipse.   \label{fig:caustic1}}
	\end{center}
\end{figure}
 \end{theorem}
 
 \begin{proof} We follow \cite{bry} to obtain the billiard map as a composition of two deck transformations.  Consider the pair of nested   ellipses parametrized by \begin{align*}
     \mathcal{E}:&\;\; f(z,w)=\frac{z^2}{a^2}+\frac{w^2}{b^2}-1=0\\
     \mathcal{E}_1:&\;\; g(x,y)= \frac{x^2}{a_c^2}+\frac{y^2}{b_c^2}-1=0.
 \end{align*}
 A tangent (oriented) line to $\mathcal{E}_1$ (caustic), passing through $q_0=(x,y)$ is given by
 \[h(x,y,z,w) =\frac{xz}{a_c^2}+\frac{yw}{b_c^2}-1=0.\]
 Now consider the set
 $\Sigma= \{(x,y,z,w): f(z,w)=g(x,y)=h(x,y,z,w)=0\}.$  
 The set $\Sigma$ is the union of two disjoint circles (curves  diffeomorphic  to circles) given by
 $\Sigma_+=\{p\in \Sigma: xw-yz>0\} $ and
  $\Sigma_-=\{p\in \Sigma: xw-yz<0\}. $
  Given $q_0\in\mathcal{E}_1$, let $p_0=(z,w)\in \mathcal{E}$ such that $(q_0,p_0)\in\Sigma_+.$ A line passing through $p_0$ and tangent to $\mathcal{E}_1$ passes through the point $q_1=(u,v)$ and $(u,v,z,w)\in\Sigma_-.$
  
  The projection  $\pi_1:\Sigma \to \mathcal{E}_1$ is a double cover. The same for the projection
  and $\pi:\Sigma\to \mathcal{E}$. 
  Now we observe that there is a unique map $\tau:\Sigma_{\pm}\to \Sigma_{\mp}$ such that $\tau(x,y,z,w)=(x,y,\bar{z},\bar{w})$. Here $(\bar{z},\bar{w})$ is the other point of intersection of the tangent line passing $(x,y)$ with the outer ellipse $\mathcal{E}.$
  
Also there is a unique map $ \sigma:\Sigma_{\pm}\to \Sigma_{\mp}$ such that $\tau(x,y,z,w)=( \bar{x},\bar{y},z,w)$. The point $q_1=( \bar{x},\bar{y})\in\mathcal{E}_1$ is the in polar line of $p_0=(z,w).$ 
Therefore the billiard orbit can be defined as follows. For each $q_i\in \mathcal{E}_1$, let $p_i\in\mathcal{E}$ the point of intersection of tangent line at $q_i$ to $\mathcal{E}_1$ meets $\mathcal{E}$ with $(q_i,p_i)\in\Sigma_+.$ Now let $q_{i+1} $ the unique point on $\mathcal{E}_1$ such that $\{q_i,q_{i+1}\}$ are on the two tangent lines to $\mathcal{E}_1$ that pass through $p_i$. Therefore,   the map $q_i\to q_{i+1}$ is given by $\sigma\circ \tau $
  (resp. $p_i\to p_{i+2}$) is an orientation preserving diffeomorphism on $\mathcal{E}_1$ (resp. on $\mathcal{E}$).

\begin{figure}[H]
	\begin{center}
		\def\svgwidth{1.0\textwidth}
		\input{pics_tex/bilharorbita.pdf_tex}
		%\includegraphics[angle=0, width=12cm]{bi.pdf}
		\caption { Three types of billiard orbits in the ellipse.   \label{fig:caustic2}}
	\end{center}
\end{figure}

When the caustic is a hyperbola is necessary to consider the  second iteration to obtain an orientation diffeomorphism. See \cite{birkhoff_1922} and \cite{kolod_1985}.

Finally, when the orbit pass through a focus the billiard map is conjugated to a diffeomorphism of the circle having two hyperbolic fixed points.
 \end{proof}
