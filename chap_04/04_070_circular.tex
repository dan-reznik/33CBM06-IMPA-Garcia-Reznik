\textcolor{red}{paper 11, family ties}

Referring to \cref{fig:04-nonconcentric-circular}:

\begin{proposition}
If a triangle center $\X\ab=\alpha X_2+ \beta X_3$ is a fixed linear combination of $X_2$ and $X_3$ for some $\alpha,\beta\in\mathbb{C}$, its locus over 3-periodics in the non-concentric pair with a circumcircle is a circle centered on $\mathcal{O}_\alpha$ and of radius $\mathcal{R}_\alpha$ given by:

\[ \mathcal{O}_\alpha = \frac{\alpha(f+g)}{3},\;\;\; \mathcal{R}_\alpha =\frac{|\alpha f g|}{3}\]
\label{prop:LinComb-concentric}

Furthermore, the center and radius of the locus do not depend on $\beta$ since the circumcenter $X_3$ is stationary at the origin of this system.
\end{proposition}

\begin{proof}
Since, $z_1,z_2,z_3$ are the 3 vertices of the Poncelet triangle inscribed in the unit circle, its barycenter and circumcenter are given by $X_2=(z_1+z_2+z_3)/3$ and $X_3=0$, respectively. We define $\X\ab:=\alpha X_2+ \beta X_3=\alpha (z_1+z_2+z_3)/3$. Using Definition~\ref{def:bla}, we get $\X\ab=\alpha(f+g+\l \ol{f}\ol{g})/3=\alpha(f+g)/3+\l(\alpha \ol{f}\ol{g})/3$, where the parameter $\l$ varies on the unit circle $\T$. Thus, the locus of $\X_{\gamma}$ over the Poncelet family of triangles is a circle with center $\mathcal{O}_{\alpha}:=\alpha(f+g)/3$ and radius $\mathcal{R}_{\alpha}:=|\alpha \ol{f}\ol{g}|/3=|\alpha f g|/3$.
\end{proof}

Using $\alpha=1-\gamma, \beta=\gamma$ for a fixed $\gamma\in\R$ in   \cref{prop:LinComb-concentric}, we get:

\begin{corollary}
 If a triangle center $\X_\gamma=(1-\gamma) X_2+ \gamma X_3$ is a real affine combination of $X_2$ and $X_3$ for some $\gamma\in\R$, its locus over 3-periodics in the non-concentric pair with a circumcircle is a circle. Moreover, as we vary $\gamma$, the centers of these loci are collinear with the fixed circumcenter.
 \label{cor:gamma-with-circumcircle}
\end{corollary}

Many triangle centers in \cite{etc} are affine combinations of the barycenter $X_2$ and circumcenter $X_3$. See   \cref{obs:affine-euler-line} for a compilation of them.

\begin{observation}
For a generic triangle, only $X_{98}$, and $X_{99}$ are simultaneously on the Euler line and on the circumcircle. However these are not linear combinations of $X_2$ and $X_3$. Still, if a triangle center is always on the circumcircle of a generic triangle (there are many of these, see \cite[Circumcircle]{mw}), its locus over 3-periodics in the non-concentric pair with circumcircle is trivially a circle.
\end{observation}

 
\begin{corollary}
 Over the family of 3-periodics inscribed in a circle and circumscribing a non-concentric inellipse centered at $O_c$, the locus of $X_k$, $k$ in 2,4,5,20 are circles whose centers are collinear. The locus of $X_5$ is centered on $O_c$. The centers and radii of these circular loci are given by:

\begin{alignat*}{4}
    O_2&=\frac{f+g}{3},\quad& O_4&=f+g,\quad&O_5&=\frac{f+g}{2},\quad&O_{20}&=-(f+g)\\
    r_2&=\frac{|f g|}{3},\quad&r_4 &= |f g|,\quad&r_5 &= \frac{|f g|}{2},\quad& r_{20}&= |f g|
\end{alignat*}

\end{corollary}

\begin{proof}
As in  \cref{cor:gamma-with-circumcircle}, we can use  \cref{prop:LinComb-concentric} with $\gamma=0,-2,-1/2,4$ to get the center and radius for $X_2,X_4,X_5,X_{20}$, respectively. All of these centers are real multiples of $f+g$, so they are all collinear. Moreover, the center $O_5$ of the circular loci of $X_5$ is $(f+g)/2$, that is, the midpoint of the foci of the inellipse, or in other words, the center $O_c$ of the inellipse.
\end{proof}
 
Referring to  \cref{fig:04-nonconcentric-circular}:

\begin{observation}
The family of 3-periodics in the pair with circumcircle includes obtuse triangles if and only if $X_3$ is exterior to the caustic. \end{observation}

This is due to the fact that when $X_3$ is interior to the caustic, said triangle center can never be exterior to the 3-periodic. Conversely, if $X_3$ is exterior, it must also be external to some 3-periodic, rendering the latter obtuse.

\begin{figure}
    \centering
    \includegraphics[width=\textwidth]{pics_03_010_n3_circumcircle_pair.eps}
    \caption{\textbf{Left:} 3-periodic family (blue) in the pair with circumcircle where the caustic contains $X_3$, i.e., all 3-periodics are acute. The loci of $X_4$ and $X_{20}$ are interior to the circumcircle. \textbf{Right:} $X_3$ is exterior to the caustic, and 3-periodics can be either acute or obtuse. Equivalently, the locus of $X_4$ intersects the circumcircle. In both cases (left and right), the loci of $X_k$, $k$ in 2,4,5,20 are circles with collinear centers (magenta line). The locus of $X_5$ is centered on $O_c$. The center of the $X_2$ locus is at $2/3$ along $O O_c$. \href{https://youtu.be/HXgJQo2UT_8}{Video}}
    \label{fig:04-nonconcentric-circular}
\end{figure} 




