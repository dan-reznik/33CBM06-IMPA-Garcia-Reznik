\section{Exercises}

\begin{exercise}
Show that over the poristic family, the locus of the foci of the $X_9$-centered circumconic (the circumbilliard) is a circle.
\end{exercise}

\begin{exercise}
Prove \cref{prop:04-antiorthic}. Furthermore, prove the intersection point of $X_1 X_3$ with the antiorthic axis is the Schröder point $X_{1155}$.
\end{exercise}

\begin{exercise}
Prove that over the poristic family the inconic centered on $X_1$ is axis-parallel with the circumconic centered on $X_9$ (i.e., the circumbilliard), see this \href{https://youtu.be/0VHBjdHXbJc}{Video}.
\end{exercise}


\begin{exercise}
Recall the cosine circle $\Cm$ (also known as the second Lemoine circle) is centered on a triangle's symmedian point $X_6$. Let $\E'$ be the Brocard ellipse of some triangle $T$. Let $\beta$ be the aspect ratio of $\E'$, i.e., $a'/b'$. Show that for any $T$, above (resp. below) a certain $\beta$, $\Cm$ is tangent to $\E'$ at two distinct points (resp. it is exterior to $\E'$). See it \href{https://bit.ly/2RqhUQV}{Live}.
\end{exercise}


\begin{exercise}
Show that the poristic excentral family is also the polar image of billiard excentrals wrt to a circle centered on a billiard (i.e., the caustic) focus. See it \href{https://bit.ly/33c1s9A}{Live}.
\end{exercise}

\begin{exercise}
Show that over the Brocard porism the radius $r^*$ of the cosine circle is invariant.
\end{exercise}

\begin{exercise}
Show that the first Lemoine circle (centered on $X_{182}$ is stationary over the Brocard porism. Above a certain $a'/b'$, this circle is tangent to one of the minor vertices of the caustic. See it \href{https://bit.ly/3tp0XUq}{Live}.
\end{exercise}

\begin{exercise}
Ehrmann's ``third'' Lemoine circle is studied in \cite{darij2012-ehrmann}, centered on $X_{576}$, is defined as follows: for each vertex, consider the 3 circles containing pairs of vertices and the symmedian point $X_6$. The third Lemoine circle contains the 6 intersections of said circles (2 each) with the sidelines. Prove this circle is also stationary over the Brocard porism, i.e., all three Lemoine circles are; see it \href{https://bit.ly/3tw09gA}{Live}. 
\end{exercise}


\begin{exercise}
Prove the expression and inequality for $\cot{\omega}$ in \cref{prop:04-brocard-w}.
\end{exercise}

\begin{exercise}
That the Brocard axis $X_3 X_6$ is stationary over the Brocard porism is established. Prove that the Lemoine axis, which intersects the Brocard axis at the Schoutte point $X_{187}$, is also stationary; see it \href{https://bit.ly/3nTRi75}{Live}.
\end{exercise}

\begin{exercise}
The so-called ``second'' Brocard triangle, defined in \cite[Second Brocard Triangle]{mw}, has vertices at the intersections of symmedians (cevians through $X_6$) with the Brocard circle. Show that over the Brocard porism, the family of second Brocard triangles is a new, smaller Brocard porism which shares the isodynamic points $X_{15}$ and $X_{16}$ with the original family. Prove that if this is iterated, the shrinking porisms converge to $X_{15}$. See it \href{https://bit.ly/3ttMNBg}{Live}.
\end{exercise}

