\section{Research Questions}

\begin{question}
Show that (i) the family of tangential triangles to the Brocard porism is also Ponceletian (caustic is the Brocard circumcircle).(ii) Derive the axes for the ellipse it is inscribed in.  and that (iii) its Gergonne point $X_7$ is stationary and coincides with the symmedian point $X_6$ of the Brocard porism; (iv) the locus of $X_{20}$ of the tangentials is a segment along the Brocard axis of the original family. \href{https://bit.ly/2RpNxdn}{Live} 
\end{question}

\begin{question}
The 3 Apollonius' circles of a triangle pass through a vertex and its two isodynamic points $X_{15}$ and $X_{16}$, see \cite[Isodynamic points]{mw}. Prove that over the Brocard porism, the sum of the inverse squared radii of the three Apollonius circles is invariant, see them \href{https://bit.ly/3elEzXI}{Live}.
\end{question}

\begin{question}
Prove that the polar image of the Brocard porism with respect to a circle centered on a caustic focus is another (tilted, smaller) Brocard porism whose Brocard inellipse shares a focus with the original one. Where does the sequence of Porisms converge? See it \href{https://bit.ly/3b7erOg}{Live}.
\end{question}

