
 In this Section we prove the locus of a given fixed linear combination of $X_2$ and $X_3$ is an ellipse. We will use Blaschke products since, as shown in \cref{fig:04-affine}, a generic non-concentric pair is always the affine image of a pair with circumcircle.
 
 \begin{figure}
    \centering
    \includegraphics[width=\textwidth]{pics_04_030_affine_circumcircle.eps}
    \caption{Affine transformation that sends a generic ellipse pair and its 3-periodic family (left) to a new pair with circumcircle (right). We parametrize the 3-periodic orbit with vertices $z_i$ in the circumcircle pair using the foci of the latter's caustic $f$ and $g$, and then apply the inverse affine transformation to get a parametrization of the vertices $P_i$ of the original Poncelet pair. \href{https://youtu.be/6xSFBLWIkTM}{Video}}
    \label{fig:04-affine}
\end{figure}

\begin{figure}
    \centering
    \includegraphics[width=.8\textwidth]{pics_04_010_n3_nonconcentric.eps}
    \caption{A pair of ellipses in general position which admits a Poncelet 3-periodic family (blue). Let the outer one be centered at the origin $O$. Their major axes are tilted by $\theta$, and their centers displaced by $O_c=(x_c,y_c)$. \href{https://youtu.be/bjHpXVyXXVc}{Video}}
    \label{fig:04-n3-nonconcentric}
\end{figure}

Consider the generic pair of nested ellipses $\E=(O,a,b)$ and $\E_c=(O_c,a_c,b_c.\theta)$ in Figure~\ref{fig:04-n3-nonconcentric}. Let s$\theta$, c$\theta$ denote the sine and cosine of $\theta$, respectively. Define $c_c^2=a_c^2-b_c^2$. The Cayley condition for the pair to admit a 3-periodic family is given by:

{\small
\begin{align}
&{b}^{4}x_c^{4}+2\,{a}^{2}{b}^{2}x_c^{2}y_c^{2}+
 \left(  2 c_c^2  \left( -{b}^{2}({a}^{2}+{b}^{2} )\right)  \text{c}\theta^2  - 2\left(  \,b ^{2}- \,b_c
^{2} \right) {b}^{2}{a}^{2}-2\,{b}^{4}b_c^{2} \right)x_c
^{2} \label{eqn:cayley}\\
&-8\,{a}^{2}{b}^{2}x_c\,{  y_c}\,c_c^2 
\text{s}\theta\text{c}\theta  +{a}^{4}y_c^{4} + \left(  2 c_c^2 a^2 \left(
{a}^{2}+{b}^{2}  \right)\text{c}\theta^2  
 -2 \left(  \,b_c^{2}+{b}^{2} \right) {a}^{4}+2
\,{a}^{2}{b}^{2}b_c^{2} \right) y_c^{2} \nonumber\\
&+ c_c^4  c^4  \left( \text{c}\theta^4-2\, c_c^2 c^2
  \left( {a}^{2} a_c^{2}-{b}^{2}{a}^{2}+
b_c^{2}{b}^{2} \right) \text{c}\theta^2 \right. \nonumber\\
 &+ \left( a a_c+a b-b b_c \right)  \left( a a_c
-a b -b b_c \right)  \left( a a_c+a b+b b_c \right)  \left( a
a_c-a b+b b_c \right) = 0\nonumber
\end{align}
}

%A related result is that the locus of the circumcenter-of-mass \cite{sergei2014-circumcenter-of-mass}, a generalization of $X_3$ for $N>3$ is also an ellipse.

Referring to Figure~\ref{fig:04-nonconcentric-xns}:

\begin{theorem}
Over the family of 3-periodics interscribed in an ellipse pair in general position (non-concentric, non-axis-aligned),
if $\X\ab$ is a fixed linear combination of $X_2$ and $X_3$, i.e., $\X\ab=\alpha X_2+\beta X_3$ for some fixed $\alpha,\beta\in\mathbb{C}$, then its locus is an ellipse. 
\label{thm:04-ellipse-locus}
\end{theorem}

\begin{proof}
Consider a general $N=3$ Poncelet pair of ellipses that forms a 1-parameter family of triangles. Without loss of generality, by translation and rotation, we may assume the outer ellipse is centered at the origin and axis-aligned with the plane $\R^2$, which we will also identify with the complex plane $\mathbb{C}$. Let $a,b$ be the semi-axis of the outer ellipse, and $a_c,b_c$ the semi-axis of the inner ellipse, as usual. 

Referring to Figure~\ref{fig:04-affine}, consider the linear transformation that takes $(x,y)\mapsto(x/a,y/b)$. This transformation takes the outer ellipse to the unit circle $\T$ and the inner ellipse to another ellipse. Thus, it transforms the general Poncelet $N=3$ system into a pair where the outer ellipse is the circumcircle, which we can parametrize using Blaschke products \cite{daepp-2019}. In fact, to get back to the original system, we must apply the inverse transformation that takes $(x,y)\mapsto(a x,b y)$. As a linear transformation from $\mathbb{C}$ to $\mathbb{C}$, we can write it as $L(z):=p z+q \ol{z}$, where $p:=(a+b)/2, q:=(a-b)/2$.

Let $z_1,z_2,z_3\in\T\subset\mathbb{C}$ be the three vertices of the circumcircle family, parametrized as in \cref{def:bla}, and let $v_1:=L(z_1),v_2:=L(z_2),v_3:=L(z_3)$ be the three vertices of the original general family. The barycenter $X_2$ of the original family is given by $(v_1+v_2+v_3)/3$, and the circumcenter $X_3$ is given by \cite{stackexchange-x3a}:

\[
    X_3=\left|
        \begin{array}{ccc}
          v_1 & |v_1|^2 & 1 \\
          v_2 & |v_2|^2 & 1 \\
          v_3 & |v_3|^2 & 1
        \end{array}
      \right| \Bigg/
     \left|
        \begin{array}{ccc}
          v_1 & \overline{v_1} & 1 \\
          v_2 & \overline{v_2} & 1 \\
          v_3 & \overline{v_3} & 1
        \end{array}
      \right|
\]

Since $\ol{z_1}=1/z_1,\ol{z_2}=1/z_2,\ol{z_3}=1/z_3$, we can write $v_1,v_2,v_3$ as rational functions of $z_1,z_2,z_3$, respectively. Thus, both $X_2$ and $X_3$ are symmetric rational functions on $z_1,z_2,z_3$. Defining $\X\ab=\alpha X_2+\beta X_3$, we have consequently that $\X\ab$ is also a symmetric rational function on $z_1,z_2,z_3$. Hence, we can reduce its numerator and denominator to functions on the elementary symmetric polynomials on $z_1,z_2,z_3$. This is exactly what we need in order to use the parametrization by Blaschke products.

In fact, we explicitly compute:
\[  \X\ab= \frac{p^2 q \left(\sigma_2 (\alpha +3 \beta )+3 \beta  \sigma_3^2\right)+\alpha  p^3 \sigma_1 \sigma_3-p q^2 (3 \beta +\sigma_1 \sigma_3 (\alpha +3 \beta ))-\alpha  q^3 \sigma_2}{3 \sigma_3 (p-q) (p+q)}\]
where $\sigma_1,\sigma_2,\sigma_3$ are the elementary symmetric polynomials on $z_1,z_2,z_3$.

Let $f,g\in\mathbb{C}$ be the foci of the inner ellipse in the circumcircle system. Using Definition~\ref{def:bla}, with the parameter $\l$ varying on the unit circle $\T$, we get:

\begin{equation}
\X\ab= u \l+v\frac{1}{\l}+w
\label{eqn:xi-param}
\end{equation}

\noindent where:

\begin{align*}
    u:=&\frac{p \left(\ol{f} \ol{g} \left(\alpha  p^2-q^2 (\alpha +3 \beta )\right)+3 \beta  p q\right)}{3 (p-q) (p+q)}\\
    v:=&\frac{\beta  p q (q-f g p)}{(q-p) (p+q)}+\frac{1}{3} \alpha  f g q\\
    w:=&\frac{q \left(\ol{f}+\ol{g}\right) \left(p^2 (\alpha +3 \beta )-\alpha  q^2\right)+p (f+g) \left(\alpha  p^2-q^2 (\alpha +3 \beta )\right)}{3 (p-q) (p+q)}
\end{align*}

By   \cref{lem:ell-param}, this is the parametrization of an ellipse centered at $w$, as desired. As in  \cref{lem:ell-param}, it is also possible to explicitly calculate its axis and rotation angle, but these expressions become very long.
\end{proof}

In \cref{thm:04-ellipse-locus} a linear combination of $X_2$ and $X_3$ was considered in terms of complex parameters $\alpha,\beta$. Below this result is specialized to the case of an affine combination of said centers in terms of a real parameter $\gamma$.

\begin{corollary}
Over the family of 3-periodics interscribed in an ellipse pair in general position (non-concentric, non-axis-aligned),
if $\X_\gamma$ is a real affine combination of $X_2$ and $X_3$, i.e., $\X_\gamma=(1-\gamma) X_2+\gamma X_3$ for some fixed $\gamma\in\R$, then its locus is an ellipse. Moreover, as we vary $\gamma$, the centers of the loci of the $\X_\gamma$ are collinear.
\end{corollary}

\begin{proof}
Apply   \cref{thm:04-ellipse-locus} with $\alpha=1-\gamma, \beta=\gamma$ to get the elliptical loci. As in the end of the proof of  \cref{thm:04-ellipse-locus}, the center of the locus of $\X_\gamma$ can be computed explicitly as 
\begin{gather*}
    w=w_0+w_1 \gamma \text{, where}\\
    w_0=\frac{1}{3} \left(q \left(\ol{f}+\ol{g}\right)+p (f+g)\right)\\
    w_1=\frac{q \left(2 p^2+q^2\right) \left(\ol{f}+\ol{g}\right)-p (f+g) \left(p^2+2 q^2\right)}{3 (p-q) (p+q)}
\end{gather*}
As $\gamma\in\R$ varies, it is clear the center $w$ sweeps a line.
\end{proof}

We proved that all of the following triangle centers have elliptic loci in the general N=3 Poncelet system, including the barycenter, circumcenter, orthocenter, nine-point center, and de Longchamps point (reflection of the orthocenter  about the circumcenter of a triangle):

\begin{observation}
Amongst the 40k+ centers listed on \cite{etc}, about 4.9k triangle centers lie on the Euler line \cite{etc-central-lines}. Out of these, only 226 are fixed affine combinations of $X_2$ and $X_3$. For $k<1000$, these amount to $X_k,k=${\small 2, 3, 4, 5, 20, 140, 376, 381, 382, 546, 547, 548, 549, 550, 631, 
632}.
\label{obs:affine-euler-line}
\end{observation}

\begin{figure}
     \centering
     \includegraphics[width=\textwidth]{pics_04_020_n3_nonconcentric_loci.eps}
     \caption{A 3-periodic is shown interscribed between two nonconcentric, non-aligned ellipses (black). The loci of $X_k$, $k=2,3,4,5,20$ (and many others) remain ellipses. Those of $X_2$ and $X_4$ remain axis-aligned with the outer one. Furthermore the centers of all said elliptic loci are collinear (magenta line). \href{https://youtu.be/p1medAei_As}{Video}}
     \label{fig:04-nonconcentric-xns}
 \end{figure}
 
%\begin{corollary}
%The elliptic loci of $X_2$, $X_3$ and $X_4$ are given by:
%\[ \textcolor{red}{mark} \]
%\textcolor{red}{these will be hard to get explicitly}
%\end{corollary}
 
\begin{observation}\label{obs:X2X4}
The elliptic loci of $X_2$ and $X_4$ are axis-aligned with the outer ellipse.
\end{observation} 

%\begin{figure}
%    \centering
%    \includegraphics[width=.7\textwidth]{pics_04_040_n3_nonconcentric_circular_caustic.eps}
%    \caption{A 3-periodic (blue) is shown inscribed in an outer ellipse and an inner non-concentric circle centered on $O_c$. The loci of both circumcenter (solid red) and Euler center (solid green) are ellipses whose major axes pass through $O_c$. \href{https://youtu.be/w7sZ5O8k4xU}{Video}}
%    \label{fig:04-circular-caustic}
%\end{figure}

Experimental evidence suggests:

\begin{conjecture}
Over 3-periodics interscribed between two ellipses in general position, the locus of a triangle center $X_k$ is an ellipse if and only if $X_k$ is a fixed linear combination of $X_3$ and $X_4$.
\label{conj:04-locus}
\end{conjecture}

