This research started upon conversations with Jair Koiller about the geometry of elliptic billiard trajectories, following one author's recent work in control of heliostat fields for solar energy plants\footnote{Such fields are in fact a discretized/flattened focusing paraboloid, see \cite{sundrop2016,esolar2017}.}; see \cite{gross2020-solar} for a recent publication. An early, naïve experimental artifact was an animation of 3-periodics in the elliptic billiard along with the locus of their incenter; see  \cite{dsr_vid11incenter}. A natural choice given that each vertex is bisected by the ellipse normal. At the time we did not know this locus could be an ellipse, and indeed, how rare a find this is (we have conjecture that amongst the 5d space of possible ellipse pairs, only in the confocal pair -- a 1d subspace -- can the locus of the incenter be a conic). A twin animation was also produced depicting the self-intersected locus of the intouchpoints; see \cite{dsr_vid11e}.

Shortly thereafter \cite{olga14} produced a proof using methods of complex algebraic geometry.
%by complexification of the phenomenon. 
This was followed by an alternative    proof using techniques of real analytic   geometry given explicitly the equation of the locus; see \cite{garcia2019-incenter}. The loci centroids of Poncelet polygons was studied in \cite{schwartz2016-com}. In  \cite{garcia2019-incenter} the centroid locus was given explicitly.  \cite{corentin2021-circum} and \cite{garcia2018} proved that the locus of the circumcenter over billiard 3-periodics is also an ellipse.

%Another related topic is the generalization of the circumcenter, i.e., the circumcenter of mass, which is independent of triangulation \cite{sergei2014-circumcenter-of-mass}. 