
Reader used to projective geometry may find it redundant to list so many Poncelet family cases. Note however that loci considered are Euclidean.

\subsection*{History of the Results}

This research started upon conversations with Jair Koiller about the geometry of elliptic billiard trajectories, following one author's recent work in control of heliostat fields for solar energy plants\footnote{Such fields are in fact a discretized/flattened focusing paraboloid, see \cite{sundrop2016,esolar2017}.}; see \cite{gross2020-solar} for a recent publication. An early, naïve experimental artifact was an animation of 3-periodics in the elliptic billiard along with the locus of their incenter \cite{dsr_vid11incenter}, a natural choice given that each vertex is bisected by the ellipse normal. At the time we did not know this locus could be an ellipse, and indeed, how rare a find this is (we have conjecture that amongst the 5d space of possible ellipse pairs, only in the confocal pair -- a 1d subspace -- can the locus of the incenter be a conic). A twin animation was also produced depicting the self-intersected locus of the intouchpoints \cite{dsr_vid11e}.

depicted an elliptic-looking locus of the incenter over billiard 3-periodics.  as well as curious loci of the intouchpoints. Over the next couple of years these served as basis for a proof by complexification of the phenomenon \cite{olga14}. One of the authors did an independent proof for the ellipticity of the incenter locus by showing its affine curvature is constant \cite{garcia2019-incenter}. In fact the elliptic locus of the centroids of Poncelet polygons had been established previously \cite{schwartz2016-com}. That the locus of the circumcenter over billiard 3-periodics was also elliptic was proved in \cite{corentin2021-circum}.  

%Another related topic is the generalization of the circumcenter, i.e., the circumcenter of mass, which is independent of triangulation \cite{sergei2014-circumcenter-of-mass}. 