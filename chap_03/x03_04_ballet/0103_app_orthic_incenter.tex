Let $T=ABC$ be any triangle and $T'=A'B'C'$ its Orthic, Figure~\ref{fig:orthic-incenter}. Let $X'_1$ be the orthic's Incenter. Referring to Figure~\ref{fig:orthic-incenter}.

\begin{lemma}
If $T$ is acute, the Incenter $X'_1$ of $T'$ is the Orthocenter $X_4$ of $T$.
\label{lem:fagnano}
\end{lemma}

\begin{proof}
This is Fagnano's Problem, i.e., the Orthic is the inscribed triangle of minimum perimeter, and the altitudes of $T$ are its bisectors \cite[Section 3.3]{rozikov2018-billiards}. Since the altitudes or $T$ are bisectors of $T'$ this completes the proof.
\end{proof}

\begin{lemma*}[\ref{lem:pinned}]
If $T$ is obtuse, the Incenter $X'_1$ of $T'$ is the vertex of $T$ subtending the obtuse angle.
\end{lemma*}

\begin{proof}
Let $B$ be the obtuse angle. Then $B'$ will be on the longest side of $T$ whereas $A'$ (resp. $C'$) will lie on extensions of $BC$ (resp. $AB$), i.e., $A'$ and $C'$ are exterior to $T$. Therefore, $X_4$ will be where altitudes $AA'$ and $CC'$ meet, also exterior to $T$. Since $AA'{\perp}CB$ and $CC'{\perp}AB$, then $CA'$ and $AC'$ are altitudes of triangle $T_e=A{X_4}C$. Since these meet at $B$, the latter is the Orthocenter of $T_e$ and $T'=A'B'C'$ is its Orthic\footnote{The pre-image of $T'$ comprises both $T$ and $T_e$.}. By Lemma~\ref{lem:fagnano}, lines $CA',AC',{X_4}B'$ are bisectors of $T'$, therefore their meetpoint $B$ is the Incenter of $T'$. 
\end{proof}

\begin{corollary}
When $T$ is obtuse, $T_e = A{X_4}C$ is acute and the Excentral Triangle of $T'$.
\end{corollary}

Since all vertices of $T'$ lie on the sides of $T_e$, this is the situation of Lemma~\ref{lem:fagnano}, i.e., $T_e$ is acute. Notice the sides of $T_e$ graze each vertex of $T'$ perpendicular to the bisectors, which is the construction of the Excentral Triangle.

Let $Q$ be a generic triangle and $Q_e$ its Excentral \cite{mw}. Let $\theta_i$ be angles of $Q$ and $\phi_i$ those of $Q_e$ opposite to the $\theta_i$'s. By inspection, $\phi_i=\frac{\pi-\theta_i}{2}$, i.e., all excentral angles are less than $\pi/2$.

\begin{corollary}
If $T$ is obtuse, $X_4$ is an Excenter of $T'$ \cite{coxeter67}.
\end{corollary}

$X_4$ is the intermediate vertex of $T_e$.
