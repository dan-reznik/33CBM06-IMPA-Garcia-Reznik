

\begin{exercise}\label{ex:31} 
Show that any triangle is an orbit of an elliptic billiard. 
\end{exercise}
 
 
 \begin{exercise}\label{ex:32} 
 The power of a point $Q$ with respect to a circle centered at $C_0$ of radius $\mu$ is given by $|Q-C_0|^2-\mu^2$ \cite[Circle Power]{mw}. Let $\mathcal{C}(t)$ be the (moving) circumcircle to the 3-periodic billiard orbits $\{P_1(t),P_2(t),P_3(t)\}$ centered at  $X_3(t)$.
 The power of the billiard center $O$ with respect to $\mathcal{C}(t)$ is invariant and equal to $-\delta$.
\end{exercise}

\begin{exercise}\label{ex:33} 
The {\em cosine circle} (also known as the second Lemoine circle) \cite[Cosine Circle]{mw} of a triangle passes through 6 points: the 3 pairs of intersections of sides with lines drawn through the symmedian $X_6$ parallel to sides of the orthic triangle. Recall that the orthic vertices are the feet of altitudes. Its center is $X_6$ \cite[Cosine Circle]{mw}. If one takes the excentral triangle of a billiard orbit as the reference triangle,   its orthic is the orbit itself.

Show that the cosine circle of the excentral triangle is invariant over the family of 3-periodic orbits. Its radius $r^*=(a^2-b^2)/\sqrt{2\delta-a^2-b^2}$ is constant and it is concentric and external to the elliptic billiard.
\end{exercise}