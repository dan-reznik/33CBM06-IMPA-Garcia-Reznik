\begin{question}
Referring to \cref{fig:03-n3-dual}, are there any conserved quantities for the dual family besides stationarity of $X_4$ at the common center?
\end{question}

\begin{question}
Referring to  the dashed green triangle in \cref{fig:03-n3-affine}(middle), are there any conserved quantities and/or fixed triangle centers for the family which is an $s$-affine image of billiard excentrals?
\end{question}

\begin{question}
Consider the family of inversive images of excentral triangles with respect to a circle centered at a point $M$ in the plane. Show the symmedian point $X_6$ of such a family will be stationary regardless of $M$. Compute the location of $X_6$. See this curious phenomenon in a \href{https://youtu.be/wwX_QfkjVi0}{Video}.
\end{question}

\begin{question}
Show that (i) the family of tangential triangles to the Brocard porism is also Ponceletian (caustic is the Brocard circumcircle).(ii) Derive the axes for the ellipse it is inscribed in.  and that (iii) its Gergonne point $X_7$ is stationary and coincides with the symmedian point $X_6$ of the Brocard porism; (iv) the locus of $X_{20}$ of the tangentials is a segment along the Brocard axis of the original family. \href{https://bit.ly/2RpNxdn}{Live} 
\end{question}

\begin{question}
The 3 Apollonius' circles of a triangle pass through a vertex and its two isodynamic points $X_{15}$ and $X_{16}$, see \cite[Isodynamic points]{mw}. Prove that over the Brocard porism, the sum of the inverse squared radii of the three Apollonius circles is invariant, see them \href{https://bit.ly/3elEzXI}{Live}.
\end{question}

\begin{question}
Prove that the polar image of the Brocard porism with respect to a circle centered on a caustic focus is another (tilted, smaller) Brocard porism whose Brocard inellipse shares a focus with the original one. Where does the sequence of Porisms converge? See it \href{https://bit.ly/3b7erOg}{Live}.
\end{question}

\begin{question}
Given a reference triangle $T$, its tangential triangle $T'$ has sides tangent to the circumcircle at the vertices of $T$. A known fact is that the sides of $T'$ are parallel to those of the orthic $T_h$ of $T$, see \cite[Tangential Triangle]{mw}. For any acute triangle $T$, the Gergonne point $X_7'$ of the tangential triangle coincides with the symmedian $X_6$ of $T$, see \cite[Contact Triangle]{mw}.

Let $T$ denote the Poncelet family of excentral triangles. We've seen above that (i) this family is all acute, and that (ii) its symmedian point $X_6$ is stationary. Let $T'$ denote their tangential triangles. This family will be non-Ponceletian: its vertices do not sweep a conic nor do its sides envelop one.

Since the $T$ are all acute, they can be thought of as the contact triangles of the $T'$. Therefore the Gergonne point $X_7'$ of the tangentials to the excentrals will coincide with $X_6$ of the excentrals and be stationary, see \cite[Contact Triangle]{mw}.

Prove that the ratio of homothety between the orthics (billiard 3-periodics) and the tangentials is invariant. Corollaries: (i) the $T'$ conserve perimeter; (ii) they conserve the same $r/R$ as the excentral orthics, i.e., the corresponding billiard 3-periodics. See it \href{https://bit.ly/3o7JM8V}{Live}.

Also prove that the locus of $X_9'$ of the $T'$ is an ellipse.

Derive equations for the curves swept by the vertices of $T'$ as well as their caustic. 
\end{question}

\begin{question}
Consider the homothetic family and its polar image with respect to a focus of the outer ellipse $\E$. Prove that (i) the caustic is a circle, derive its location and radius. (ii) the family is inscribed in a conic, namely, below (resp. above) a certain aspect ratio $a/b$ of $\E$, the conic is an an ellipse (resp. hyperbola). (iii) the Gergonne point $X_7$ of the family is stationary.
Live: family inscribed in \href{https://bit.ly/33p7xj6}{ellipse}, \href{https://bit.ly/3bbTaTt}{hyperbola}.
\end{question}