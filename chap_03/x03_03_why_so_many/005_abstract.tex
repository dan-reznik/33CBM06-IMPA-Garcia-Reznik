\begin{abstract}
A triangle center such as the incenter, barycenter, etc., is specified by function on sidelengths and/or angles. Consider the 1d family of 3-periodics in the elliptic billiard, and the loci of its triangle centers. Some will sweep ellipses, others quartics, sextics, etc. We propose two rigorous methods to prove if the locus of a given center is an ellipse. The first one is based on computer algebra, and the second one on the Theory of Resultants, which yields an implicit two-variable polynomial whose zero set contains the locus. We also prove that if the triangle center function is rational on sidelengths, the locus is algebraic.\\

\noindent \textbf{Keywords}: elliptic billiard, periodic trajectories, triangle center, derived triangle, locus, loci, algebraic.\\

\noindent \textbf{MSC2010} {37-40 \and 51N20 \and 51M04 \and 51-04}

\end{abstract}

 
 