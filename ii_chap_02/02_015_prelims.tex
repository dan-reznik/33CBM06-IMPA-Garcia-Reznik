\section{Preliminaries}

Consider a pair of nested ellipses $\mathcal{D}\subset \mathcal{C}$ as shown in \cref{fig:poncelet}. Let $P_0\in \mathcal{C}$ and draw a tangent line $L_1$ to $\mathcal{D}$ and passing through $P_0$. Call $P_1$ the second intersection of $L_1$ with $\mathcal{C}.$
From $P_1$ draw the the second tangent line $L_2$ to $\mathcal{D}.$
Clearly this process can be iterated to order $n. $
The sequence of points $\{P_0,P_1,P_2,\ldots, P_n,\ldots\}$ will be called the {\em Poncelet orbit}.

When $P_n=P_0$ the Poncelet orbit is called periodic and the polygon $\mathcal{P}_n$ with vertices $\{P_0,\ldots, P_{n-1}, P_n\}$ will be called an $n-$gon. So, we obtain a polygon interscribed in the pair of ellipses $\{\mathcal{D},\mathcal{C}\}.$