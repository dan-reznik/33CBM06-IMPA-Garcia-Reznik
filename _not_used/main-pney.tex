\documentclass{book}

% For a set of fonts a bit better than standard TeX:
\input{setup}

\begin{document}

\title{The title goes here}
\author{John Doe, III}
\maketitle

\frontmatter

\tableofcontents

\mainmatter

% Use \include{} and \input always to segregate the files of each chapter.

\chapter{The first chapter}
  \label{chap1}
  \section{Entering the text}
    \label{sec1.1}

Enter the text as you would normally.

The encoding of the files should be UTF-8 and that should be set in the text editor.
This should be followed by all files including bibliography, etc.

  \section{Using labels for references}

Place \verb+\label{}+ in all chapters, sections, \dots so you can refer to all elements
using the commands \verb+\cref{}+ from the package \texttt{cleveref}.

\begin{theorem}[Euclid, c. 300 BC] 
	\label{euclid}
	There are infinitely many primes.
\end{theorem} 

And this is a reference to \cref{euclid} of \cref{sec1.1} of \cref{chap1}.

    \subsection{}

      \paragraph{}

\chapter{The use of a foreign language}

\section{How to mark it}

All words not in English should be marked appropriately, so if the proof
is by \textlatin{reductio ad absurdum} it should be marked as such.

The word \emph{eigenvector} is derived from the German``\textgerman{Eigenvektor}''.

% And if you have a text that is slightly longer
%
%\begin{greek}[variant=ancient]
%μῆνιν ἄειδε θεὰ Πηληϊάδεω Ἀχιλῆος οὐλομένην, ἣ μυρί' Ἀχαιοῖς ἄλγε'
%ἔθηκε, πολλὰς δ' ἰφθίμους ψυχὰς Ἄϊδι \index[std]{discriminant}προί̈αψεν ἡρώων, αὐτοὺς δὲ ἑλώρια
%τεῦχε κύνεσσιν οἰωνοῖσί τε πᾶσι, Διὸς δ' ἐτελείετο βουλή, ἐξ οὗ δὴ τὰ
%πρῶτα διαστήτην ἐρίσαντε Ἀτρεί̈δης τε ἄναξ ἀνδρῶν καὶ δῖος Ἀχιλλεύς.
%\end{greek}

\chapter{Creating indices}

  \section{Itema that should go to the Nomenclature Index}

The \emph{discriminant} of this base is the
element of $K$ \nomenclature[]{$\Delta(\omega_1, \ldots, \omega_n)$}{discriminant of the base $\omega_1, \ldots, \omega_n$}.

$L$\nomenclature[]{$L$}{subring of $K$}

\section{Items going to the other indices}

John Milnor\index[aut]{Milnor, John} was the first to show that there are 
several differentiable structures\index[std]{differentiable structures} on
the same manifold\index[std]{manifold}, in particular on the sphere
$\mathbb{S}^7$\index[not]{$\mathbb{S}^7$}.


\chapter{Another showing bibliographical citations}

The system of citations give you three basic commands that produce:

% paulo ney: use textcite[]{}

{\small
\begin{verbatim}
  \cite{jon90}                   -->    Jones et al. 1990
  \cite[cap. 2]{jon90}           -->    Jones et al. 1990, cap. 2
  \textcite{jon90}               -->    Jones et al. (1990)
  \textcite[cap. 2]{jon90}       -->    Jones et al. (1990, cap. 2)
  \parencite{jon90}              -->    (Jones et al., 1990)
  \parencite[cap. 2]{jon90}      -->    (Jones et al., 1990, cap. 2)
  \parencite[see][]{jon90}       -->    (see Jones et al., 1990)
  \parencite[see][cap. 2]{jon90} -->    (see Jones et al., 1990, cap. 2)
\end{verbatim}
}

If you only open one set of square brackets it will assume the
contents of the brackets is a \emph{postnote}, so if you only 
want a \emph{prenote} make sure you still open the second set
of square brackets and then just leave it empty.

One of the most useful commands is \verb+\textcite+ which marks
the citation clearly and let you sew the citation with the text,
as in the phrase:

\begin{quote}
	The problems with the command \verb+\eqnarray+ in \LaTeX{} were
	listed in \textcite{madsen}.
\end{quote}

Several other commands are available to cite the author name only, or the 
year, \dots
\begin{verbatim}
    \citeauthor
    \citeyear
    \citetitle
\end{verbatim}
for a more harmonious and precise integration with the text.

\section{How to colect references}

By far one of the best ways to collect references is to use the AMS \emph{MRef}
on the URL:

\begin{quote}
\url{https://mathscinet.ams.org/mref}
\end{quote}

Almost all are comple and very close to an ideal reference,
neding only a few transformations.

\chapter{Labelling graphics}

The best way to add labels to a graphics is to use \TeX{}
to manage all tags -- text and placement. We gain with a 
precise choice of the location of the label, typography that
is the same as the text and the possibility of using elaborate
labels with a mathematical basis.

This is even more important when the graphics are produced by
other programs that do not offer positioning and tag control.

The figure below is drawn by a simulation in the program Mathematica.
The labels here are prepared by the program that despite the precision
the figure has no precise labeling:

\begin{center}
  \includegraphics[width=0.7\textwidth]{fig_with_labels}
\end{center}

As you can see the figure labels are produced in a different source
from the rest of the book, the $X2$, $X3$ and $X20$ labels are literally on top
ellipses, and the positioning of the indexes of $P_1$, $P_2$ and
$P_3$ have a lower quality typographic placement.

The best way is to produce the figure without labels, as in:

\begin{center}
  \includegraphics[width=0.7\textwidth]{fig_no_labels}
\end{center}

and then add the labels with TikZ, as in:

\begin{center}
\begin{tikzpicture}
  \node (image) at (0,0) {
            \includegraphics[width=0.7\textwidth]{fig_no_labels}
    };
  \node [blue, font=\bfseries] at (4.3,1.9) {$P_1$};
  \node [blue, font=\bfseries] at (1,3.7) {$P_2$};
  \node [blue, font=\bfseries] at (-2.1,-0.5) {$P_3$};

  \node [brown, font=\bfseries] at (1.1,1.7) {\tiny{$X2$}};
  \node [red, font=\bfseries] at (1.4,0) {\small{$X3$}};
  \node [black, font=\bfseries] at (0.7,5.1) {\small{$X4$}};
  \node [green, font=\bfseries] at (1,2.7) {\scriptsize{$X5$}};
  \node [teal, font=\bfseries] at (1.99,-5) {\small{$X20$}};
\end{tikzpicture}
\end{center}

The position of the labels above were obtained by trial and error, but
there are some tools to help position the labels.

The use of \emph{TikZ} allows the use of colors, more elaborate typography,
including mathematical equations, arrows and more complex annotations.

\appendix

\chapter{The first appendix}

\backmatter

\addcontentsline{toc}{chapter}{Bibliography}
\nocite{*}
\sloppy
\printbibliography

\printindex[not]
\printindex[aut]
\printindex[std]

\printnomenclature

\end{document}
