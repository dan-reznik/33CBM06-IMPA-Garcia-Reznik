\section{Main Result}

History of experiment. Jair referenced a book \cite{daepp-2019} on whatsapp, not knowing of our work on loci. Mark quickly learned Blaschke products and noticed its usefulness for 3-periodic Poncelet.

\begin{theorem}
Over 3-periodics in Poncelet pair (concentric or not) with a circumcircle, the locus of a triangle center which is a fixed linear combination of $X_3$ and $X_4$ is a circle given by:
 
\[ ... \]

Furthermore all locus centers are collinear with the origin.
\end{theorem}

\begin{theorem}
Over 3-periodics interscribed in a concentric, axis-aligned pair of ellipses, the power of the common center with respect to either the circurmcircle or Euler's circle is invariant and given by:

\[ ... \]
\end{theorem}


Mark.

\section{Properties of Brocard porism triangles}
\label{sec:review}
\input{chap5/020_review}

\section{Properties of the family of second Brocard triangles}
\label{sec:broc-second}
\input{chap5/030_second_brocard}

\section{An infinite sequence of porisms}
\label{sec:porism-seq}
\input{chap5/040_porism_sequence}

\section{Embedding the discrete sequence of porisms in a continuous family}
\label{sec:continuous}
\input{chap5/050_continuous_family}

\section{Future Work}

