We adopt Kimberling's $X_k$ notation for triangle centers, e.g., $X_1$ for the incenter, $X_2$ for the barycenter, etc. \cite{etc}. Assume that $X_3=(0,0)$ is at the origin. Referring to Figure~\ref{fig:brocard-basic}

\begin{definition}[Brocard Points]
Let a reference triangle have vertices $ABC$ when traversed counterclockwise. The first Brocard point $\Omega_1$ is where sides $AB$, $BC$, $CA$ concur when rotated a special angle $\omega$ about $A$, $B$, $C$, respectively. The second Brocard point $\Omega_2$ is defined similarly by a $-\omega$ rotation $CB$, $BA$, $AC$ about $C$, $B$, $A$, respectively.
\end{definition}

\noindent Referring to Figure~\ref{fig:broc-por-sec-tri}:

\begin{definition}[Brocard Circle]
This is the circle $\K$ through $X_3$, and the Brocard points $\Omega_1, \Omega_2$.
\end{definition}

\noindent Note $\K$ also contains $X_6$, and it is centered on the midpoint $X_{182}$ of $X_3 X_6$ \cite{mw}.\\

\noindent It turns out a 1d family of Poncelet triangles can be constructed with constant Brocard angle \cite{bradley2007-brocard}. Referring to Figure~\ref{fig:broc-por-sec-tri}:

\begin{definition}[Brocard Porism]
This is a 1d family of Poncelet triangles inscribed in a fixed circle of radius $R$, and circumscribed about a special ellipse known as the Brocard inellipse $\E$, whose foci are on the Brocard points.
\end{definition}

\noindent Note also that since $\E$ is stationary, so are its foci, i.e., the Brocard points $\Omega_1$ and $\Omega_2$ of the family. Also, since the family is inscribed in a fixed circle, its circumcenter $X_3$ is motionless.


\begin{corollary}
Over the Brocard porism the Brocard circle and $X_6$ are stationary.
\end{corollary}

\noindent This stems from the fact that over the porism $\Omega_1,\Omega_2,X_3$ are stationary. Since $X_6$ is antipodal to $X_3$ on the so-called Brocard axis \cite{mw}, it is also stationary.

%\begin{figure}
%    \centering
%    \includegraphics[width=\textwidth]{pics/0020_second_brocard_circle.eps}
%    \caption{Given a reference triangle (blue), the vertices $A_2B_2C_2$ of the Second Brocard Triangle (purple) are the non-$X_6$ intersections of cevians through $X_6$ (dashed purple) with the Brocard circle (orange). Notice the latter contains both Brocard points $\Omega_1$ and $\Omega_2$, $X_3$ and $X_6$. The latter two are antipodes and $X_1X_3$ (the Brocard axis) is perpendicular to $\Omega_1 \Omega_2$. Also shown (olive green) is the Brocard inellipse, centered at the Brocard midpoint $X_{39}$ with foci on the Brocard points.}
%    \label{fig:sec-broc}
%\end{figure}

After \cite{etc}:

\begin{definition}[Barycentric Combo]
Let $P$ and $U$ be finite points on a triangle's plane with normalized barycentrics $(p,q,r)$ and $(u,v,w)$, respectively. Let $f$ and $g$ be homogeneous functions of the sidelengths. The $(f,g)$ {\em combo} of $P$ and $U$, also denoted $f*P + g*U$, is the point with barycentrics $(f\,p + g\,u,f\,q + g\,v, f\,r + g\,w)$.
\end{definition}

\noindent The Cyrillic letter $\Sha$ shall henceforth denote $\cot\omega$. Since $\omega\in[0,\pi/6]$ \cite[Brocard Angle]{mw}, then:

\begin{remark}
 $\Sha{\geq}\sqrt{3}$.
 \label{rem:sha-min}
\end{remark}

\begin{lemma}
Over the porism, the two Isodynamic points $X_{15}$ and $X_{16}$ are stationary and given by:

\[
X_{15}=\left[0, \frac{R(\sqrt{3}-\Sha)}{
 \sqrt{\Sha^2-3}}\right],\;\;\;
  X_{16}=\left[0, -\frac{R(\sqrt{3}+\Sha)}{
 \sqrt{\Sha^2-3}}\right]
\]
\label{lem:x15x16}
\end{lemma}

\begin{proof}
Peter Moses (cited in \cite[X(15), X(16)]{etc}) derives the following combos for the two isodynamic points:

\begin{align}
X_{15} =& \sqrt{3}*X_3 + \Sha*X_6 \label{eqn:combo-x15} \\
X_{16} =& \sqrt{3}*X_3 - \Sha*X_6 \nonumber
\end{align}

With all involved quantities invariant, the result follows. Note that isodynamic points are self-inverses with respect to the circumcircle  \cite[Isodynamic Points]{mw}. For how we obtained the explicit expressions see Appendix~\ref{app:app_further}.
\end{proof}

\begin{lemma}
Let $R$ and $\omega$ denote a triangle's circumradius and Brocard angle. The semi-axes $(a,b)$ and center $X_{39}$ of the Brocard inellipse $\E$ are given by:

\begin{gather*}
[a,b]= R\left[\sin\omega,2\sin^2\omega\right]=R\left[\frac{1}{\sqrt{1+\Sha^2}},\frac{2}{{1+\Sha^2}}\right]\\
 X_{39}=\left[0,-\frac{R\Sha\sqrt{\Sha^2-3}}{\Sha^2+1}\right]
\end{gather*}
 \label{lem:ab}
\end{lemma}

\begin{proof}

Consider a triangle $T$ with sidelengths $s_1,s_2,s_3$, area  $\Delta$, and circumradius  $R$. The following identities appear in \cite{bradley2007-brocard,shail1996-brocard}:

\[R=\frac{s_1 s_2 s_3}{4\Delta}, \;\;\; \sin\omega=\frac{2\Delta}{\sqrt{\lambda}},\;\;\; |\Omega_1-\Omega_2|^2=4c^2=4R^2\sin^2\omega (1-4\sin^2\omega)\]

\noindent where $\lambda=(s_1 s_2)^2+(s_2 s_3)^2+ (s_3 s_1)^2$ (named $\Gamma$ in \cite[Eqn. 2]{shail1996-brocard}), and $c^2=a^2-b^2$. The result follows from combining the above into the following expressions for the Brocard inellipse axes \cite[Brocard Inellipse]{mw}:

\[ a =\frac{s_1 s_2 s_3}{2\sqrt{\lambda}},\;\;\;\;\;\; b =\frac{2 s_1 s_2 s_3 \Delta}{\lambda}.\]

\noindent For how explicit expressions were obtained for $X_{39}$, see Appendix~\ref{app:app_further}.

\end{proof}

\begin{proposition}
The circumradius $R$ and $\Sha$ are given by:

 \[R=\frac{2 a^2}{b},\;\;\; \Sha=\frac{\sqrt{4 a^2-b^2}}{b}.\]
 \label{prop:wRab}
\end{proposition}

\begin{proof}
Follows directly from Lemma \ref{lem:ab}.
%This follows from the well known equation \cite[Brocard angle]{mw}
% \[\cot\omega=\frac{s_1^2+s_2^2+s_3^2}{4\Delta}\]
% where $s_i$ are  the sidelengths of $T$ and $\Delta $ is its area.
\end{proof}

\begin{lemma}
The coordinates for the symmedian point $X_6$ are given by:

\[
X_6=\left[0,-\frac{R\sqrt{ \Sha^2 -3}}{ \Sha}\right] \]
\label{lem:x6}
\end{lemma}

\noindent A derivation is provided in Appendix~\ref{app:app_further}.

\begin{corollary} The distance between circumcenter and symmedian point is given by

\[ |X_3-X_6|=\frac{R\sqrt{ \Sha^2 -3}}{ \Sha}\]
\end{corollary}

\begin{proposition}
In terms of $R$ and $\Sha$, the Brocard points are given by:

 \[\Omega_{1,2}(R,\Sha)=\frac{R\sqrt{ \Sha^2-3}}{ \Sha^2+1} \left[\pm1, - \Sha  \right]\]
 \label{prop:brocs12}
\end{proposition}

\begin{proof} Follows from Proposition \ref{prop:pair_brocard} and Lemma \ref{lem:reciprocal}.
\end{proof}

\begin{corollary}
The center $X_{39}$ of the Brocard inellipse $\E$ is given by:
\[ X_{39}=\left[0,-R {\frac {\Sha\,
\sqrt {\Sha^2-3}}{\Sha^2+1}} \right]\]
\end{corollary}





