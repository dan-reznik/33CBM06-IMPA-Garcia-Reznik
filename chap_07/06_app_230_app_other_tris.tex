Let $T_k$ denote the nth-Brocard triangle, $k=1,\cdots,7$, as defined in  \cite{gibert2020-brocard}. Referring to Figure~\ref{fig:broc-por-tris}

\begin{remark}
For $k=1,2,5,7$, the locus of vertices of $T_k$ trace out the Brocard circle.
\end{remark}

This stems from the fact that by construction, $T_1$, $T_2$, $T_7$ are inscribed in the Brocard circle (their circumcenter is $X_{182}$ and $T_5$ is homothetic to the reference one and its circumcenter is $X_{9821}$ \cite{etc}. For no other Poncelet families and/or Brocard triangle combinations have we been able to identify conic loci for the Brocard Triangle vertices.

\begin{figure}
    \centering
    \includegraphics[width=\textwidth]{pics/0090_broc_porism_tris.eps}
    \caption{A Poncelet triangle (blue) in the Brocard Porism is shown inscribed in a circle and circumscribed aboutthe Brocard inellipse (both black). Its Brocard points $\Omega_1$ and $\Omega_2$ are stationary at the foci of said inellipse. The First, Second, and Seventh Brocard Triangle are shown inscribed in the Brocard circle (green). The Fifth Brocard Triangle (orange)is homothetic about the Poncelet triangle and therefore its locus will also be a circle (not shown). \href{bit.ly}{Video}}
    \label{fig:broc-por-tris}
\end{figure}

\begin{observation}
Both $\Omega_1^1$ and $\Omega_2^1$
move along the same circle $C_1$.
\end{observation}

\begin{observation}
Both $\Omega_1^6$ and $\Omega_2^6$
move along the same circle $C_6$.
\end{observation}

\begin{observation}
The locus of $\Omega_1^4$ and $\Omega_2^4$ are two distinct circles $C_4$ and $C_4'$.
\end{observation}

\section{ Results}

Consider a triangle $T$ with vertices $A=[-a,0]$, $B=[a,0]$ and
$C=[0,b]$.

Then the Brocard points of $T$ are given by

\[\Omega_1=\left[   \frac{  a( 3\,a^{2}-b^{2})}{9\,a^{2}+b^{2}}, \, \frac {4{a}^{2}b
}{9\,{a}^{2}+{b}^{2}}
\right],  \;\;\;\Omega_2=\left[ - {\frac{  a( 3\,a^{2}-b^{2})}{9\,{a}^{2}+{b}^{2}}}, \,{\frac {4{a}^{2}b
}{9\,{a}^{2}+{b}^{2}}}
\right] \]

The cotangent of the Brocard angle of $T$ is
\[\cot\omega=\frac{3a^2+b^2}{2ab}\]

Let $X_6=\left[ 0,\frac{ 2 a^2b}{ 3 a^2+b^2}\right]$ be the triangular center of $T$.

Then the points $A, X_6, \Omega_2$ and $B, X_6, \Omega_1$ are aligned.



The second Brocard triangle $T_2$ have vertices

\[ A=\Omega_1, \; B=\Omega_2, \; C=X_3=\left[0,\frac{b^2-a^2}{2b}\right]\]

The Brocard points of $T_2$ are 
\begin{align*}
\Omega_1'=&\left[  \frac{ a(3 a^2-b^2)^3}{(9 a^4+42 a^2 b^2+b^4) (9 a^2+b^2)}, \frac{-12 a^2 b (3 a^4-14 a^2 b^2-b^4)}{(9 a^4+42 a^2 b^2+b^4) (9 a^2+b^2)}\right]\\
 \Omega_2'=&\left[  \frac{-a(3 a^2-b^2)^3}{(9 a^4+42 a^2 b^2+b^4) (9 a^2+b^2)}, \frac{-12 a^2 b (3 a^4-14 a^2 b^2-b^4)}{(9 a^4+42 a^2 b^2+b^4) (9 a^2+b^2)}   \right]
\end{align*}


The cotangent of the Brocard angle $\omega_2$ of $T_2$ is
\[\cot\omega_2= 
 \frac {9\,{a}^{4}+18\,{a}^{2}{b}^{2}+{b}^{4}}{4ab \left( 3\,{a}^{
2}+{b}^{2} \right) }\]


The second Brocard triangle $T_2'$ have vertices

\[ A=\Omega_1', \; B=\Omega_2', \; C=X_3'=\left[0,\frac{-3 a^4+6a^2b^2+b^4}{4b(3a^2+b^2)} \right]\]
The Brocard points of $T_2'$ are 
\begin{align*}
\Omega_1''=&\left[a''  ,b'' \right]\\
a''=& -{\frac { \left( 3\,{a}^{2}-{b}^{2} \right) ^{7}a}{ \left( 9\,{a}^{4}+
42\,{a}^{2}{b}^{2}+{b}^{4} \right)  \left( 9\,{a}^{2}+{b}^{2} \right) 
 \left( 81\,{a}^{8}+1620\,{a}^{6}{b}^{2}+1206\,{a}^{4}{b}^{4}+180\,{a}
^{2}{b}^{6}+{b}^{8} \right) }}\\
b''=&- \,{\frac { 4{a}^{2} \left( 135\,{a}^{6}-387\,{a}^{4}{b}^{2}-75\,{a}^{2
}{b}^{4}-{b}^{6} \right)  \left( 27\,{a}^{6}+189\,{a}^{4}{b}^{2}+105\,
{a}^{2}{b}^{4}+7\,{b}^{6} \right) b}{ \left( 9\,{a}^{4}+42\,{a}^{2}{b}
^{2}+{b}^{4} \right)  \left( 9\,{a}^{2}+{b}^{2} \right)  \left( 81\,{a
}^{8}+1620\,{a}^{6}{b}^{2}+1206\,{a}^{4}{b}^{4}+180\,{a}^{2}{b}^{6}+{b
}^{8} \right) }}
\\
\Omega_2''=&\left[ -a''  ,b''
 \right]
\end{align*}


The cotangent of the Brocard angle $\omega_2'$ of $T_2'$ is
\[\cot\omega_2'=  \,{\frac {81\,{a}^{8}+756\,{a}^{6}{b}^{2}+630\,{a}^{4}{b}^{4}+84\,{
a}^{2}{b}^{6}+{b}^{8}}{8ab \left( 9\,{a}^{4}+18\,{a}^{2}{b}^{2}+{b}^{4}
 \right)  \left( 3\,{a}^{2}+{b}^{2} \right) }}
  \]
The circles containing the family of Brocard points $\Omega_1,\Omega_2',\Omega_1'', \ldots$ points (respec. $\Omega_2,\Omega_1',\Omega_2'',\ldots $) are given by

\begin{align*}
    C_1:\;&\left( 3\,{a}^{2}-{b}^{2} \right) (x^2+y^2) -2\,a \left( {a}^{2}+{b}^{2} \right) x-8\,{a}^{2}b y   -
 \left( {a}^{2}-3\,{b}^{2} \right) {a}^{2} = 0
\\
    C_2:\;&\left( 3\,{a}^{2}-{b}^{2} \right) (x^2+y^2) +2\,a \left( {a}^{2}+{b}^{2} \right) x-8\,{a}^{2}b y   -
 \left( {a}^{2}-3\,{b}^{2} \right) {a}^{2} = 0
\end{align*}

The centers  of $C_1$ and $C_2$ are \textcolor{red}{check it}
  \[ O_1=\left[\frac{a(a^2+b^2)}{3a^2-b^2},\frac{ 4a^2b}{3a^2-b^2}\right],\;\;
O_2=\left[-\frac{a(a^2+b^2)}{3a^2-b^2},\frac{ 4a^2b}{3a^2-b^2}\right]
\]
The intersections of   $C_1$ and $C_2$ are 

\[ \left[0,\frac { a (\sqrt {3}({a}^{2}+ {b}^{2})+4\,ab )  }{3\,
{a}^{2}-{b}^{2}}
 \right], \;\;\; \left[0,- \frac { a (\sqrt {3}({a}^{2}+ {b}^{2})-4\,ab )  }{3\,
{a}^{2}-{b}^{2}}\right]
\]