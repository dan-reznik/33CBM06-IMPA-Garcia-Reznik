\section{Exercises}
\label{sec:07-exercises}

\begin{exercise}
Consider a cubic polynomial $p(z)=(z-\alpha_1)(z-\alpha_2)(z-\alpha_3)$ with simple roots $\alpha_i$ (i=1,2,3).
Let $\beta_1$ and $\beta_2$ the roots of $p'(z)$.
Consider the family of confocal ellipses having foci $\beta_1$ and $\beta_2$.

Show that there exists a unique ellipse $\mathcal E$ in this family passing through the midpoints $(\alpha_i+\alpha_j)/2$, and that it is tangent to the sides of the triangle $T=\{\alpha_1,\alpha_2,\alpha_3\}$. This ellipse is known as Steiner innelipse of $T$.

Conclude that the center of $\E$ is the triangular center  $X_2$ of $T$ and that $T$ is a 3-periodic orbit of a homothetic Poncelet pair.


\end{exercise}

\begin{exercise} Consider an ellipse $\E$ and the set of tangent lines. Show that the set of points of intersection between any two perpendicular
tangents to $\E$  lie on a circle. Find the radius and the center of this circle.
\end{exercise}

\begin{exercise}
Consider a circle $\mathcal{C}$ and a point $P_0$.   Consider the family of circles passing through $P_0$ and internally tangent to $\mathcal{C}$. Show that the set of centers of this family of circles is an ellipse.
Find the semi-axes and the foci of the ellipse.
\end{exercise}

\begin{exercise}
In the proof of \cref{prop:07-X1q2}, let $z_1(\lambda)$, $ z_2(\lambda) $ and $z_3(\lambda)$ the roots of   $E_2(z,\lambda)=0$, $\lambda \in \mathbb{T}$. Show that  the trace of these three curves is an ellipse, i.e., they parametrize the excentral locus.   
\end{exercise}