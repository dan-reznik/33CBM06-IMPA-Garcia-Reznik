In the previous chapter we restricted ourselves to observing various interesting locus-related phenomena in the confocal pair. 
Namely, we attempt to answer the following related questions:

\begin{itemize}
    \item When are loci algebraic?
    \item What is the type of curve for the locus of the incenter (and excenters) in a generic concentric, axis-parallel (CAP) pair?
    \item In \cite{sergei2016-com} it is shown that over Poncelet N-periodics in a generic pair of conics, the locus of vertex and area centroids is an ellipse. For triangles, both points collapse to $X_2$. Given 3-periodics in a generic pair of ellipses, when is the locus of some triangle center an ellipse? To answer we will use technique based on Blaschke products \cite{daepp-2019}, review below;
    \item We consider the special case of 3-periodics in a pair with a circumcircle, showing that many such loci are circles.
\end{itemize}