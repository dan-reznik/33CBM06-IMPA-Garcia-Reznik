

\begin{definition}
A congruence of lines in the plane is a 1d parameter of straight lines.
\end{definition}

 
  % \index[std]{congruence of line}
   In the cartesian form, a congruence is given by:
 

\begin{align}  R(x,y,u)=a(u)x+b(u)y+c(u)=0,\;\; a(u)^2+b(u)^2\ne  0
\label{eq:appB-cr}
\end{align}
It will be assumed that the congruence is of class  $C^k,\, k\geq 3$. 

In general, the equation   $R(x,y,u)=0$ defines a regular surface in   $\mathbb{R}^3$  and its projection   $\pi(x,y,u)=(x,y)$ in the plane is singular in the set  

  \[\mathcal{C}= \{(x,y,u): \; R(x,y,u)=  R_u(x,y,u) =0 \} \]
  which is called the   \textit{criminant set}.
  The projection $\pi(\mathcal{C})$ is called the \textit{envelope} of the congruence. It is also called the     \textit{  discriminant.}
%\index[std]{criminant}
%\index[std]{discriminant}
%\index[std]{envelope}

Te envelope is given by:

\begin{equation}\label{eq:env}
x(u)=\frac{b c^\prime-c b^\prime}{a b^\prime-b a^\prime},\;\;\; y(u)=\frac{c a^\prime-a c^\prime}{a b^\prime-b a^\prime}.
\end{equation}

 

\begin{example}  The envelope of the family of lines defined by   $x \cos u\;  +y\sin u \;  =h(u)$ is the curve
	$E(u)=h(u)(\cos u,\sin u)	+h^\prime(u)(-\sin u, \cos u). $
\end{example}

 
A congruence of lines, in the parametric form, is given by
 
\[ x=x_0(t)+v a(t), y=y_0(t)+v b(t)\]

Therefore, the envelope is given by:
 
	
	\begin{align*}
  x(t)=&x_0(t)+\frac{   a( b x'_0-a y'_0)}{a b^\prime-b a^\prime} \\
	y(t)=&y_0(t)+ \frac{   b( b x'_0-a y'_0)}{a b^\prime-b a^\prime} 	\end{align*}
 

 
  For more on theory of envelopes see  \cite[Chapter  3]{arnold-1994},  \cite[pp. 305]{berger-1992}, \cite[Chapter   5]{bruce-1992}. %\cite[Chap{ghys-2017}.  % \cite{kling}. 
 

 

