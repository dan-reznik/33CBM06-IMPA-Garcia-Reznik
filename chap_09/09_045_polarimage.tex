

We review properties of polar and pedal transformations. A detailed treatment can be found in \cite{akopyan2007-conics,stachel2019-conics}.

In the discussion that follows, all geometric objects are contained in a fixed plane. Let $\Cm$ be a circle centered at $f_1$. The polar transformation with respect to $\Cm$ maps each straight line not passing through $f_1$ into a point, and maps each point different from $f_1$ into a straight line. This is done in the following manner.

 Let $p\neq f_{1}$ be a point and let ${p}^\dagger$ be the inversion of $p$ with respect to $\Cm$. The straight line $L_{p}$ that passes through ${p}^\dagger$ and is orthogonal to the line joining $p$ and ${p}^\dagger$ is the polar of $p$ with respect to $\Cm$. Conversely, a line $L$ not passing through $f_1$ has a point $q$ as its pole with respect to $\Cm$ if $L=L_{q}$.

For a smooth curve $\gamma$ not passing through $f_1$, we can define the polar curve $\gamma^{\star}$ in two equivalent ways. Let $p$ be a point of $\gamma$ and $T_{p}\gamma$ the tangent line to $\gamma$ at $p$, we define $p^{\star}=(T_{p}\gamma)^{\star}$, and $\gamma^{\star}$ is the curve generated by $p^{\star}$ as $p$ varies along $\gamma$. We can also think of $\gamma^{\star}$ as the curve that is the envelope of the 1-parameter family of lines $L_{p}$, where $p$ is a point of $\gamma$.

The notion of a polar curve can be naturally extended to polygons in the following manner: let $L_{j}$, $j=1,2,...,N$ be the consecutive sides of a planar polygon $\P$, and let $q_{j}$ be the corresponding poles, then this indexed set of points are the vertices of what we call the polar polygon $\P^{\star}$. Alternatively, we can consider the polars of vertices of $\P$, and their consecutive intersections do define the vertices of $\P^{\star}$.

 

\begin{lemma}
Let $\E$ be an ellipse and $f_1$ one its the foci. Then the polar curve $\E^{\star}$ with respect to a circle $\Cm$ centered at $f_1$ is a circle. Let $\Hc$ be a hyperbola and $f_1$ one of its foci. Then the polar curve $\Hc^{\star}$ with respect to a circle $\Cm$ centered at $f_1$ is a circle minus two points.
\label{lem:polar}
\end{lemma}
\begin{proof}

We will use polar coordinates for our computations. Without loss of generality, let $f_{1}=(0,0)$ and consider the parametrized conic given by:

\[ \gamma(t)=\left[\frac{a(1-e^2)}{1+e\cos{t}}\cos{t}, \frac{a(1-e^2)}{1+e\cos{t}}\sin{t}\right] \]

\noindent where, if $e>1$ the trace of $\gamma$ is a hyperbola and if $e<1$ the trace of $\gamma$ is an ellipse. The expression for the polar curve $\gamma^{\star}(t)$ is obtained by direct computation: compute the unit normal $\mathbf{n}(t)$ to $\gamma(t)$, and  the distance $d(t)$ from the tangent line through $\gamma(t)$ to $f_{1}$. This yields:
 
\[ \gamma^{\star}(t)=\left[\frac{e+\cos{t}}{a(1-e^2)},\frac{\sin{t}}{a(1-e^2)}\right] \]

\noindent whose trace is clearly contained in a circle. For the hyperbola, the parameter $t$ is such that $1+e\cos(t) \neq 0 $, this is why $\Hc^{\star}$ is a circle minus two points.
  
\end{proof}     


\begin{lemma}
\label{lem:limit-focus}
Let $\E_{1}$ and $\E_{2}$ be two confocal ellipses and $\E_{1}^{\star}$ and $\E_{2}^{\star}$ be the circles as in Lemma \ref{lem:polar}, then $f_{1}$ is a limiting point of the pencil of circles defined by $\E_{1}^{\star}$ and $\E_{2}^{\star}$. In a similar way, let $\E$ and $\Hc$ be respectively an ellipse and hyperbola that are confocal, and let $\E^{\star}$ and $\Hc^{\star}$ be the circle and the circle minus 2 points, as in Lemma \ref{lem:polar}, then $f_{1}$ is a limiting point of the pencil of circles defined by $\E^{\star}$ and the circle that contains $\Hc^{\star}$.
\end{lemma}

\begin{proof}
Given two circles $\Cm_{1}$ and $\Cm_{2}$, a classical result states that the limiting points $\delta_{\pm}$ of the pencil of circles determined by $\Cm_{1}$ and $\Cm_{2}$ are such that the inversion of $\Cm_{1}$ and $\Cm_{2}$ with respect to circles centered on $\delta_{\pm}$ are concentric.

If we denote by $a_{i}$ and $e_{i}$, $i=1,2$, respectively, the semi-major diameter and eccentricity of the ellipses $\E_{1}$ and $\E_{2}$, then, by symmetry, we can define an unknown limiting point as $\delta_{p}=(x,0)$, and the concentric circle condition then becomes a quadratic equation in the variable $x$, where the coefficients depend on $a_{1}$, $a_{2}$, $e_{1}$ and $e_{2}$. Using the fact that $\E_{1}$ and $\E_{2}$ are confocal, which is equivalent to $e_{1}a_{1}=e_{2}a_{2}$, and with some algebraic manipulations, the quadratic can be written as:

\[ (a_{1}-a_{2})(a_{1}+a_{2})(e_{2}a_{2}x+1)=0 \]

\noindent so $f_{1}=(0,0)$ is indeed one of the limiting points of the pencil of circles.
\end{proof}  


Under the polar transformation an origin-centered ellipse $\E$ is sent to the circle:

\[ \left( x+{\frac {c \left( {b}^{2}+\rho^{2} \right) }{{b}^{2}}}
 \right) ^{2}+{y}^{2}-{\frac {{a}^{2}\rho^{4}}{{b}^4}}=0\]
 and the confocal ellipse $\E_c$ is sent to
 \[ \left( x+\frac {c \left( b_c^{2}+\rho^{2} \right) }{b_c^2}
 \right) ^{2}+{y}^{2}-{\frac {a_c^{2}\rho^{4}}{b_c^4}}=0\]
 
Therefore:

\begin{align*} r&=\frac{a \,\rho^2}{b^2},\; R=\frac{a_c\, \rho^2}{b_c^2}\\
d&=\frac {c \left( b_c^{2}+\rho^{2} \right) }{b_c^2}-\frac {c \left( b^{2}+\rho^{2} \right) }{b^2}=\frac{c \rho^2(b^2 - b_c^2)}{b^2\, b_c^2}=\frac{\rho^2 c\, (a^2 - a_c^2)}{b^2\, b_c^2}\\
  k^2&=\frac{4Rd}{(R+d)^2-r^2}=\frac{4 \,c\, a_c\,(a_c - c)^2    }{b_c^4}\\
  \delta_{\pm}&=\pm {\frac {\rho^{6}{a}^{4}}{2\,c\, {b}^{8}}}+{\frac { \left( {a}^{2} b_c^{4}-a_c^{2}{b}^{4}-{c}^{6} \right) \rho^{2}}{2\,{b}^{2}  b_c^{2} {c}^{3}}}
\end{align*}

 The limiting points $\ell_1,\ell_2$ are given by: $[-c,0]$ and $[-c+\frac{\rho^2}{c},0]$.
 
 
 \textcolor{red}{Ron: calculos abaixos foram simplificados na secao seguinte}
 
 
 Reciprocally, given a pair of circles $C_R:\; (x+d)^2+y^2=R^2$ and $C_r:\; x^2+y^2=r^2$, by the  polar transformation, 
 it is associated a pair of confocal ellipses $\{\E, \E_c\}$
 with semiaxes given by:
 
 \begin{align*}
     a&= \sqrt{b^2+c^2}=\frac{\sqrt{2}\rho^2\sqrt{ (R^2 - d^2 - r^2)\sqrt{\alpha} +(R^2 - d^2)^2  - r^2(2R^2 - r^2) }}{2r\alpha} \\
     b&= {\frac {\sqrt {2}{\rho}^{2}}{2\,\alpha\,r}\sqrt {\alpha\, \left( 
\alpha+\sqrt {4\,\alpha\,{r}^{2}{d}^{2}+{\alpha}^{2}} \right) }}
 \\
     a_c&= \sqrt{b_c^2+c^2} = \frac{\sqrt{2}\rho^2\sqrt{(R^2 + d^2 - r^2)\sqrt{\alpha} + R^2(R^2 - 2r^2) + (d^2 - r^2)^2 }}{2R\alpha}\\
     b_c&= {\frac {\sqrt {2}{\rho}^{2}}{2 \alpha\,R}\sqrt {\alpha\, \left( \alpha+
\sqrt {4\,\alpha {R}^{2}\,{d}^{2}+{\alpha}^{2}} \right) }}
 \\
     c&= {\frac {d{\rho}^{2}}{2\,\alpha\, \left( R^2-r^2 \right)    } \left( \sqrt {\alpha\, \left( 4\, {r}^{2} d^2+\alpha
 \right) }+\sqrt {\alpha\, \left( 4\,{R}^{2}{d}^{2}+\alpha \right) }
 \right) }
\\
     \alpha&=(R - d + r)(R + d + r)(R - d - r)(R + d - r)
 \end{align*}
 
 Under the polar transformation an origin-centered ellipse $\E$ is sent to the circle:

\[ \left( x+{\frac {c \left( {b}^{2}+\rho^{2} \right) }{{b}^{2}}}
 \right) ^{2}+{y}^{2}-{\frac {{a}^{2}\rho^{4}}{{b}^4}}=0\]
 and the confocal ellipse $\E_c$ is sent to
 \[ \left( x+\frac {c \left( b_c^{2}+\rho^{2} \right) }{b_c^2}
 \right) ^{2}+{y}^{2}-{\frac {a_c^{2}\rho^{4}}{b_c^4}}=0\]
 
Therefore:

\begin{align*} r&=\frac{a \,\rho^2}{b^2},\; R=\frac{a_c\, \rho^2}{b_c^2}\\
d&=\frac {c \left( b_c^{2}+\rho^{2} \right) }{b_c^2}-\frac {c \left( b^{2}+\rho^{2} \right) }{b^2}=\frac{c \rho^2(b^2 - b_c^2)}{b^2\, b_c^2}=\frac{\rho^2 c\, (a^2 - a_c^2)}{b^2\, b_c^2}\\
  k^2&=\frac{4Rd}{(R+d)^2-r^2}=\frac{4 \,c\, a_c\,(a_c - c)^2    }{b_c^4}\\
  \delta_{\pm}&=\pm {\frac {\rho^{6}{a}^{4}}{2\,c\, {b}^{8}}}+{\frac { \left( {a}^{2} b_c^{4}-a_c^{2}{b}^{4}-{c}^{6} \right) \rho^{2}}{2\,{b}^{2}  b_c^{2} {c}^{3}}}
\end{align*}

 The limiting points $\ell_1,\ell_2$ are given by: $[-c,0]$ and $[-c+\frac{\rho^2}{c},0]$.
 
 
 \section{ Hyperbolas}
 
 Consider the pair of circles $x^2+y^2=r^2$
 and $(x+d)^2+y^2=R^2$ and the limit points
 $\ell_1= (R^2 - d^2 - r^2 -\Delta)/(2d)$ and $\ell_2=(R^2 - d^2 - r^2 + \Delta)/(2 d)$,
where
 \[ \Delta=\sqrt { \left( d+R+r \right)  \left( R-d+r \right)  \left( R+d-r
 \right)  \left( R-d-r \right) }\]

 \begin{lemma} The polar image of the circle $x^2+y^2=r^2$ with respect to the limit point $  \ell_2 $ is the hyperbola  centered at
 \[ \left[\frac{\Delta^2-2d^2k^2  }{2 d \Delta} + \frac{R^2 - d^2 - r^2}{ 2 d },0\right] \]
 and semiaxes given by
 
 \begin{align*}
     a^2&=\frac{k^4(   2d^2r^2-\Delta(R^2 - d^2 - r^2 - \Delta))}{2 r^2 \Delta^2}\\
     b^2&= \frac{k^4(R^2 - d^2 - r^2 - \Delta)}{2 \Delta r^2  } \\
     c^2&=a^2+b^2=\frac{k^4d^2}{\Delta^2}
 \end{align*}
 \end{lemma}
 
 
 \begin{lemma} The polar image of the circle $(x+d)^2+y^2=R^2$ with respect to the limit point $  \ell_2 $ is the hyperbola  centered at
 \[ \left[\frac{\Delta^2-2d^2k^2  }{2 d \Delta} + \frac{R^2 - d^2 - r^2}{ 2 d },0\right] \]
 and semiaxes given by
 
 \begin{align*}
     a^2&= \frac{k^4(2R^2d^2 - \Delta (R^2 + d^2 - r^2 - \Delta))}{ 2R^2 \Delta^2} \\
     b^2&=  \frac{(R^2+  d^2 - r^2 - \Delta)k^4}{ 2 \Delta R^2} \\
     c^2&=a^2+b^2=\frac{k^4d^2}{\Delta^2}
 \end{align*}
 \end{lemma}


\begin{lemma} The polar image of the circle $x^2+y^2=r^2$ with respect to the limit point $  \ell_1 $ is the ellipse  centered at
 \[ \left[ \frac{ 2 d^2 k^2 - \Delta^2}{ 2 \Delta d} + \frac{R^2 - d^2 - r^2}{ 2 d},0\right] \]
 and semiaxes given by
 
 \begin{align*}
     a^2&=\frac{( (\Delta + R^2 - d^2 - r^2)\Delta + 2d^2r^2)k^4}{ 2 \Delta r^2} \\
     b^2&=\frac{    (R^2+ \Delta-d^2 - r^2 ) k^4}{2 \Delta r^2} \\
     c^2&=a^2-b^2=\frac{k^4d^2}{\Delta^2}
 \end{align*}
 \end{lemma}
 
 
 \begin{lemma} The polar image of the circle $(x+d)^2+y^2=R^2$ with respect to the limit point $  \ell_1 $ is the ellipse centered at
 \[ \left[ \frac{ 2 d^2 k^2 - \Delta^2}{ 2 \Delta d} + \frac{R^2 - d^2 - r^2}{ 2 d},0\right] \]
 and semi-axes given by
 
 \begin{align*}
     a^2&= \frac{(2R^2d^2 + \Delta (  \Delta+R^2 + d^2 - r^2 ))k^4}{ 2 \Delta R^2}  \\
     b^2&=  \frac{(\Delta+R^2 + d^2 - r^2   )k^4}{ 2 \Delta R^2} \\
     c^2&=a^2-b^2=\frac{k^4d^2}{\Delta^2}
 \end{align*}
 \end{lemma}

 